\section{Program Model and Abstract Domain}\label{sec:domains}
%-------------------------------------------------------------------------------
\subsection{Program Representation}\label{sec:program}  
%
We consider \emph{bounded programs} with safety
properties given as a set of assertions, $\assertions$, in the program.
%
A bounded program is obtained by a transformation that unfolds
loops and recursions a finite number of times. The result 
is represented by a set
$\constraints=\program\cup\{\neg \bigwedge_{\assertion\in\assertions} \assertion\}$,
where $\program$ contains an encoding of the statements in the program as
constraints, obtained after translating the program
into single static assignment (SSA) form via a data flow analysis.
%
The representation $\constraints$ for the program in Fig.~\ref{fig:example1} is 
%
\begin{equation}\label{eq:ssa}
\begin{array}{l@{}l}
\{ & g_0 = (0 \leq v \leq N),\:
     g_1 = (g_0 \wedge c),\:
     x_0 = v,\:
     x_1 = -v, \\ 
  &  x_2 = g_1?x_0:x_1, \:
     g_2 = (g_1 \vee g_0\wedge \neg c), \:
     z  = x_2{\cdot}x_2,\:
     g_2 \wedge z{<}0 \}
\end{array}
\end{equation}
%
Assignments such as x:=v become equalities $x_1=v$, where the
left-hand side variable gets a subscripted fresh name.
%
Control flow is encoded using guard variables, e.g.~$g_1=g_0\wedge c$.
%
Data flow joins become conditional expressions, e.g.~$x_3=g_1?x_1:x_2$.
%
The assertions $\assertions$ are constraints such as $g_2
\Rightarrow z\geq 0$, meaning that if $g_2$ holds
(i.e.,~the assertion is reachable) then the assertion must hold.
%
We~write $\vars$ for the set of variables occurring in $\constraints$.  
Based on the above program representation, we define a \textit{safety formula}
($\formula$) as the conjunction of everything in $\constraints$, that is,  
$\formula:= \bigwedge_{\constraint\in\constraints} \constraint$. The formula 
$\formula$ is unsatisfiable if and only if the program is safe.
%
\begin{table}[t]
\small
\begin{center}
{
\begin{tabular}{c@{\quad}|@{\quad}c@{\quad}|@{\quad}c@{\quad}|@{\quad}c@{\quad}|@{\quad}c}
Interval & Octagons & Zones & Equality & Fixed-coef.~Polyhedra \\ \hline
$a \leq x_i \leq b$ & $\pm x_i \pm x_j \leq d$ & $x_i - x_j \leq d$ & 
  $x_i=x_j\Omit{,(x_i \neq x_j)}$ & $a_1x_1 + \ldots + a_nx_n \leq d$
\end{tabular}
}
\end{center}
\caption{Template instances in the template polyhedra domain}
\label{domain}
\end{table}
%
%-------------------------------------------------------------------------------
\subsection{Abstract Domain}
%-------------------------------------------------------------------------------
%Silva et al.~\cite{sas12,dhk2013-popl} showed that SAT solvers operate on the domain 
%of partial assignments~\cite{sas12,dhk2013-popl}.  
In this paper, we instantiate ACDLP over a reduced product domain~\cite{CC79}
$\domain[\vars]=\booldomain^{|\boolvars|}\times\reldomain[\numvars]$ where
$\booldomain$ is the Boolean domain that permits abstract values
$\{\true,\false,\bot,\top\}$ over boolean variables $\boolvars$ in the
program, and $\reldomain$ is a \textit{template polyhedra}~\cite{vmcai05}
domain over the numerical (bitvector) variables $\numvars$.  Our template
polyhedra domain can express various relational and non-relational templates
over $\numvars$, as given in Table~\ref{domain}.
% 
%-------------------------------------------------------------------------------
\subsubsection{Template Polyhedra Abstract Domain}
%-------------------------------------------------------------------------------
%
An abstract value of the template polyhedra domain~\cite{vmcai05}
represents a set $\numconcval$ of values of the vector $\vec{\numvar}$ 
of numerical (bitvector) variables $\numvars$ of their respective
data types. (Currently, signed and unsigned integers are supported.)
For example, in the program given by Eq.~(\ref{eq:ssa}), we have four 
numerical variables, written as the vector $\vec{\numvar} = (x_0,x_1,x_2,z)$.  
An abstract value is a constant vector $\vec{\numabsval}$ that represents 
sets of values for $\vec{\numvar}$ for which 
$\mat{C}\vec{\numvar}\leq\vec{\numabsval}$, for a fixed coefficient 
matrix $\mat{C}$.  The domain containing $\vec{\numabsval}$ is augmented 
by a special element $\bot$ to denote the minimal element of the lattice.  
%
There are several optimisation-based 
techniques~\cite{vmcai05} for computing the domain operations, 
such as meet ($\meet$) and join ($\sqcup$), in the template polyhedra domain.  
In our implementation, we use the strategy iteration approach of~\cite{BJKS15}.
%
The abstraction function is defined by $\alpha(\numconcval) = \min \{\vec{\numabsval}\mid
\mat{C}\vec{\numvar}\leq\vec{\numabsval}, \vec{\numvar}\in \numconcval\}$, where 
$\min$ is applied component-wise.  The concretisation $\gamma(\vec{\numabsval})$ is the set $\{\vec{\numvar}\mid
\mat{C}\vec{\numvar}\leq\vec{\numabsval}\}$ and $\gamma(\bot)=\emptyset$,
i.e., the empty polyhedron.

For notational convenience we will use conjunctions of linear
inequalities, for example $x_1\geq 0 \wedge x_1-z\leq 30$, to write the
abstract domain value $\vec{\numabsval}=\vecv{0}{30}$,
with $\mat{C}=\qmat{-1}{0}{1}{-1}$ and $\vec{\numvar}=\vecv{x_1}{z}$. 
\rmcmt{$\true$ corresponds to abstract value $\top$ and $\false$ to 
abstract value $\bot$}.
%
For a program with $N=|\numvars|$ variables, the template 
matrix $\mat{C}$ for the interval domain $\intervals[\numvars]$, 
has $2N$ rows. Hence, it generates at most $2N$ inequalities, one
for the upper and lower bounds of each variable.
%
For octagons $\octagons[\numvars]$, we have at most $2N^2$
inequalities, one for the upper and lower bounds of each variable and
sums and differences for each pair of variables. 
Unlike a non-relational domain, a relational domain such as octagons 
requires the computation of a \emph{closure} in order to obtain a normal 
form, necessary for precise domain operation. 
The closure computes all implied domain constraints.  
An example of a closure computation for octagonal inequalities is
$\mathit{closure}((x-y \leq 4) \wedge (y-z \leq 5))=((x-y \leq 4) \wedge (y-z
\leq 5) \wedge (x-z \leq 9))$.
%
For octagons, closure is the most critical and expensive operator; it has  
cubic complexity in the number of program variables.  We therefore compute 
closure lazily in template polyhedra domain in our abstract model search 
procedure, \rmcmt{which is described in section~\ref{lazyclosure}.} 
%We refer the reader to Appendix~\ref{appendix:lazyclosure} 
%for details of lazy closure computation. 
%-------------------------------------------------------------------------------
\subsubsection{Abstract Transformers}
%
An abstract transformer $\abstrans{\domain}{\constraint}$ transforms an
abstract value $\absval$ through a constraint $\constraint$; it
\emph{deduces} information from $\absval$ and $\constraint$.  The best
transformer is
%
\begin{equation}\label{eq:abstrans}
\abstrans{\domain}{\constraint}(\absval)=\absval\meet\alpha(\{\val\mid \val\in\gamma(\absval), \val\models \constraint\})
\end{equation}
 where we write 
$\val\models \constraint$ if the concrete value $\val$ satisfies the constraint $\constraint$.
%
Any abstract transformer that over-approximates the best abstract
transformer is a sound transformer and can be used in our algorithm.
%
For example, we can deduce $\abstrans{\domain}{x=2(y+z)}(\absval)=(0\leq y
\leq 2 \wedge\allowbreak 1 \leq y-z \leq 1\wedge\allowbreak -2\leq x\leq 6)$
for the abstract value $\absval=(0\leq y \leq 2 \wedge\allowbreak 1 \leq y-z
\leq 1)$.
%
We~denote the set of abstract transformers for a safety formula
$\formula$ using the abstract domain $\domain$ by
$\abstransset=\{\abstrans{\domain}{\constraint}\mid
\constraint\in\constraints\}$.
%
\rmcmt{
\begin{definition}
  An Abstract Conflict Graph (ACG) is a directed acyclic graph in which the 
  vertices are defined by all deduced elements or a decision node and a special 
  conflict node $(\bot)$.  The edges in ACG are obtained from the abstract
  transformer of each deduced element: if $\abstrans{\domain}{\constraint}(a) = 
  u$, where $u$ is an unique deduction from $\abstrans{\domain}{\constraint}$
  and $a$ is an initial abstract value, then there is a directed edge from each 
  elements of $a$, whose variable set has non-empty intersection with the variable 
  set of $\constraint$, to $u$. 
\end{definition}
}



%-------------------------------------------------------------------------------
\subsection{Properties of Abstract Domains}
%-------------------------------------------------------------------------------
%
An important property of a clause-learning SAT solver is that each
non-singleton element of the partial assignment domain can be 
decomposed into a set of \textit{precisely complementable} singleton
elements~\cite{dhk2013-popl}.  This property of domain elements
are necessary to learn elements that help to guide the model search away from the
conflicting region of the search space.  
%
Most numerical abstract domains, such as intervals and octagons lack
complements in general, i.e., not every element in the domain has a
precise complement.  However, these domain elements can be represented
as intersections of half-spaces, each of which admit a precise complement.
We formalise this in the sequel.
%
\begin{definition} 
A \emph{meet irreducible} $m$ in a complete lattice 
structure $A$ is an element with the following property.
\begin{equation}
\forall m_1, m_2 \in A: m_1 \meet m_2 = m \implies (m = m_1 \lor m = m_2), m \neq \top  
\end{equation}
\end{definition}
%
\rmcmt{The meet irreducibles in the Boolean domain $\booldomain$ 
for a variable $x$ are $x$ and $\neg x$. The meet 
irreducibles in the template polyhedra domain are all elements 
that concretise to half-spaces, i.e., they can be represented 
by a single inequality. For the interval domain, these are 
$x \leq d$ or $x \geq d$ for constants $d$. (Refer thesis definition)}
%
\begin{definition}
A \emph{meet decomposition} $\decomp(\absval)$ of an abstract
element $\absval \in \domain$ is a set of meet irreducibles $M \subseteq \domain$ such that 
$\absval=\bigsqcap_{m\in M} m$.
\end{definition}

\noindent For polyhedra this intuitively means that each polyhedron can be
written as an intersection of half-spaces.
%
For example, the meet decomposition of the interval domain element
$decomp(2\leq x\leq 4 \wedge 3\leq y\leq 5)$ is
the set $\{x\geq 2, x\leq 4, y\geq 3, y\leq 5\}$.
%
\begin{definition} 
An element $\absval\in \domain$ is called \emph{precisely complementable}
iff there exists $\bar{\absval} \in \domain$ such that 
$\neg\gamma(\bar{\absval})=\gamma(\absval)$.
That is, there is an element whose complemented concretisation equals
the concretisation of $\absval$.
\end{definition}
%
The precise complementation property of a partial assignment lattice can
be generalised to other lattice structures. 
%
For example, the precise complement of a meet irreducible $(x \leq 2)$ in
the interval domain over integers is $(x \geq 3)$, or the precise complement
of the meet irreducible $(x+y \leq 1)$ in the octagon domain over integers
is $(x+y \geq 2)$.  Our domain implementation supports precise
complementation operation.  However, standard abstract interpretation does
not require a complementation operator.  Hence, abstract domain libraries,
such as APRON~\cite{apron}, do not provide it.  But it can be implemented
with the help of a meet decomposition as explained above.
%
