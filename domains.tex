\section{Program Model and Abstract Domain}\label{sec:domains}
%This section introduces the program model and identifies properties of abstract
%domains that are necessary for learning in abstract lattice structures.
%\rmcmt{A formal definition of program}
% \subsection{Programs:}
% %
% A {\em program} $\program$ is formally defined as follows.
% \[
% \begin{array}[t]{@{}lll}
% \program & {:}{:}{=} & \mathit{Procedure} \\
% \mathit{Procedure} & {:}{:}{=} & \mathit{Statement} \mid \mathit{Procedure} \\
% \mathit{Statement} & {:}{:}{=} & CStatement \mid \mathit{function(Var_1,\dots,Var_n)} \\
% \mathit{CStatement} & {:}{:}{=} & x {:}{=} exp \mid \mathit{ITE}(b, s_1,s_2) \mid \mathit{s_1;s_2} \mid loop\{s\} \\
% \end{array}
% \]
% Consider sets of expressions $Exp$ and Boolean expressions $BExp$
% over variables $Var$ of $\program$.  The variables in $Var$ 
% can take numeric values in $Val$.  A procedure is denoted as 
% $\mathit{function(Var_1,\dots,Var_n)}$.  A $\mathit{CStatement}$ 
% is an assignment, conditional, sequential concatenation or a loop. \\
% \textbf{Control-flow Graph:} A CFG is a triple $(Loc, E, lbl)$, 
% where $Loc$ is the set of locations with an unique start 
% location $(init)$ and error location $(err)$, $E$ is the set 
% of control-flow edges that are labelled with the set 
% $lbl \in Statement$.  For the purpose of illustration, we 
% assume that all the procedures are inlined.  

%-------------------------------------------------------------------------------
\subsection{Program Representation}\label{sec:program}  
\Omit {
\begin{wrapfigure}{r}{3cm}
\[
\begin{array}{rcl}
g_0 &=& \true \\
\multicolumn{3}{l}{0 \leq v\leq N} \\
g_1 &=& g_0\wedge c\\
x_0 &=& v \\
x_1 &=& -v \\
x_2 &=& g_1?x_0:x_1 \\
g_2 &=& g_1 \vee g_0\wedge \neg c\\
z &=& x_2 * x_2 \\
\multicolumn{3}{l}{z<0 \wedge g_2}
\end{array}
\]
\caption{\label{fig:ssa}
SSA for the example in Fig.~\ref{fig:example}
}
\end{wrapfigure}
}
%
\Omit{
\begin{figure}[t]
\center
\scriptsize
\begin{tabular}{l}
\hline
 SSA \\
\hline
\begin{minipage}{4.50cm}
$\begin{array}{rcl}
\constraints:= (g_0 = (0 \leq v\leq N)) \wedge \\
(g_1 = (g_0 \wedge c)) \wedge (x_0 = v) \wedge \\
(x_1 = -v) \wedge (x_2 = g_1?x_0:x_1) \wedge \\
(g_2 = (g_1 \vee g_0\wedge \neg c)) \wedge \\
(z  = x_2 * x_2) \wedge (g_2 \wedge z<0)
\end{array}$
\end{minipage}
\\
\hline
\end{tabular}
\caption{Static Single Assignment Form}
\label{ssa}
\end{figure}
}
%
We consider \emph{bounded programs} with safety
properties given as a set of assertions, $\assertions$, in the program.
%
A bounded program is obtained by a transformation that unfolds
loops and recursions a finite number of times. The result 
is represented by a set
$\constraints=\program\cup\{\neg \bigwedge_{\assertion\in\assertions} \assertion\}$,
where $\program$ contains an encoding of the statements in the program as
constraints, obtained after translating the program
into single static assignment (SSA) form via a data flow analysis.
%
%Fig.~\ref{ssa} gives the SSA constraints for the program in Fig.~\ref{fig:example}.
The representation $\constraints$ for the program in Fig.~\ref{fig:example1} is 
\Omit{
\begin{equation}\label{eq:ssa}
\begin{array}{l}
(g_0{=}(0 \leq v{\leq}N)) \wedge 
(g_1{=}(g_0 \wedge c)) \wedge (x_0{=}v) \wedge 
(x_1{=}-v) \wedge \\
(x_2{=}g_1?x_0:x_1) \wedge 
(g_2{=}(g_1 \vee g_0\wedge \neg c)) \wedge 
(z {=}x_2{\cdot}x_2) \wedge (g_2 \wedge z{<}0)
\end{array}
\end{equation}
}
%\tmcmt{Rajdeep: sorry, the above needs to be a \textit{set}, not a formula. See LeTeX source.}
\begin{equation}\label{eq:ssa}
\begin{array}{l@{}l}
\{ & g_0 = (0 \leq v \leq N),\:
     g_1 = (g_0 \wedge c),\:
     x_0 = v,\:
     x_1 = -v, \\ 
  &  x_2 = g_1?x_0:x_1, \:
     g_2 = (g_1 \vee g_0\wedge \neg c), \:
     z  = x_2{\cdot}x_2,\:
     g_2 \wedge z{<}0 \}
\end{array}
\end{equation}
%
Assignments such as x:=v become equalities $x_1=v$, where the
left-hand side variable gets a subscripted fresh name.
%
Control flow is encoded using guard variables, e.g.~$g_1=g_0\wedge c$.
%
Data flow joins become conditional expressions, e.g.~$x_3=g_1?x_1:x_2$.
%
The assertions $\assertions$ are constraints such as $g_3
\Rightarrow z\geq 0$, meaning that if $g_3$ holds
(i.e.,~the assertion is reachable) then the assertion must hold.
%
We~write $\vars$ for the set of variables occurring in $\constraints$.  
Based on the above program representation, we define a \textit{safety formula}
($\formula$) as the conjunction of everything in $\constraints$, that is,  
$\formula:= \bigwedge_{\constraint\in\constraints} \constraint$. The formula 
$\formula$ is unsatisfiable if and only if the program is safe.
%
%
%For example, figure~\ref{swssa} presents the program analysis constraints for 
%a C program in Static Single Assignment (SSA) form.  

% \subsection{Concrete Semantics}
% \pscmt{not sure that adds anything for TACAS}
% The {\em concrete} domain is a lattice 
% of concrete environments, $Env = Var \rightarrow Val$, and is 
% defined by $CDom = (Env, \sle_{C}, \join_{C}, \meet_{C})$.
% A transformer, $post_{stmt}$, of concrete domain defines 
% the effect of a statement $stmt$ on the concrete domain, 
% $post_{stmt} : \powerset(Env) \rightarrow \powerset(Env)$.  
% 
% A state in concrete domain is a tuple $\langle l, \sigma \rangle$, 
% where $l$ is a location and $\sigma \in Env$.  A trace is a sequence 
% of states $(l_0, \sigma_0), \ldots (l_n, \sigma_n)$ such that for all 
% $0 \leq i \leq n$, there exists a cfg egde $(l_{i}, l_{i+1})$ 
% if $\sigma_{i+1} \in post_{i}(\sigma_{i})$. 

%-------------------------------------------------------------------------------
%\paragraph{Safety problem}  
%

% \subsection{Static Analysis Equations for Safety}
% Static program analysis based on abstract interpretation~\cite{DBLP:conf/emsoft/Cousot07} 
% perform safety analysis by computing fixed point to infer invariants 
% over program variables.  However, bounded model checking (BMC) tries to search 
% for a counterexample in a bounded execution trace, by symbolically executing a 
% program up to a finite bound.  
% Similar to BMC, ACDCL searches for a counterexample by solving the formula shown below.
% For a set $N$ of program analysis constraints defined over a set 
% of constraint variables $Var = \{X_i| i \in N\}$, representing a program
% $\program$ and a concrete domain $\powerset(Env)$, the static analysis equation 
% for safety of $\program$ is given as follows.  Here, $X_{init}, X_{Err}$ denote the initial valuation 
% and the \rmcmt{final valuation} of constraint variables.
% \[\formula = X_{init} \subseteq Env \wedge \underset{p,q \in |\program|}
% {\bigwedge} \{ X_p \subseteq post_{(p,q)}(X_q) \} \wedge X_{err} \supset \emptyset \] 

%-------------------------------------------------------------------------------
\subsection{Abstract Domain}
%-------------------------------------------------------------------------------
%Silva et. al.~\cite{sas12,dhk2013-popl} showed that SAT solvers operate on the domain 
%of partial assignments~\cite{sas12,dhk2013-popl}.  
In this paper, we instantiate ACDLP over a reduced product domain~\cite{CC79}
$\domain[\vars]=\booldomain^{|\boolvars|}\times\reldomain[\numvars]$ where
$\booldomain$ is the Boolean domain that permits abstract values
$\{\true,\false,\bot,\top\}$ over boolean variables $\boolvars$ in the
program, and $\reldomain$ is a \textit{template polyhedra}~\cite{vmcai05}
domain over the numerical (bitvector) variables $\numvars$.  Our template
polyhedra domain can express various relational and non-relational templates
over $\numvars$, as given in Table~\ref{domain}.
% 
%In our prototype implementation and for illustrative purposes 
%-------------------------------------------------------------------------------
\subsubsection{Template Polyhedra Abstract Domain}
%-------------------------------------------------------------------------------
%
An abstract value of the template polyhedra domain~\cite{vmcai05}
represents a set $\numconcval$ of values of the vector $\vec{\numvar}$ 
of numerical (bitvector) variables $\numvars$ of their respective
data types. (Currently, signed and unsigned integers are supported.)
For example, in the program given by Eq.~(\ref{eq:ssa}), we have four 
numerical variables, written as the vector $\vec{\numvar} = (x_0,x_1,x_2,z)$.  
An abstract value is a constant vector $\vec{\numabsval}$ that represents 
sets of values for $\vec{\numvar}$ for which 
$\mat{C}\vec{\numvar}\leq\vec{\numabsval}$, for a fixed coefficient 
matrix $\mat{C}$.  The domain containing $\vec{\numabsval}$ is augmented 
by a special element $\bot$ to denote the minimal element of the lattice.  
%
\begin{table}[t]
\small
\begin{center}
{
\begin{tabular}{l|l|l|l|l}
\hline
Interval & Octagons & Zones & Equality & Fixed-coefficient Polyhedra \\ \hline
$(a \leq x_i \leq b)$ & $(\pm x_i \pm x_j \leq d)$ & $(x_i - x_j \leq d)$ & 
  $(x_i=x_j)\Omit{,(x_i \neq x_j)}$ & $(a_1x_1 + \ldots + a_nx_n \leq d)$ \\ 
\hline
\end{tabular}
}
\end{center}
\caption{Templates Instances in the Template Polyhedra Domain}
\label{domain}
\end{table}
%
There are several optimisation-based 
techniques~\cite{vmcai05} for computing the domain operations, 
such as meet ($\meet$) and join ($\sqcup$), in the template polyhedra domain.  
In our implementation, we use the strategy iteration approach of~\cite{BJKS15}.
%
The abstraction function is defined by $\alpha(\numconcval) = \min \{\vec{\numabsval}\mid
\mat{C}\vec{\numvar}\leq\vec{\numabsval}, \vec{\numvar}\in \numconcval\}$, where 
$\min$ is applied component-wise.  The concretisation $\gamma(\vec{\numabsval})$ is the set $\{\vec{\numvar}\mid
\mat{C}\vec{\numvar}\leq\vec{\numabsval}\}$ and $\gamma(\bot)=\emptyset$,
i.e., the empty polyhedron.

For notational convenience we will use conjunctions of linear
inequalities, for example $x_1\geq 0 \wedge x_1-z\leq 30$, to write the
abstract domain value $\vec{\numabsval}=\vecv{0}{30}$,
with $\mat{C}=\qmat{-1}{0}{1}{-1}$ and $\vec{\numvar}=\vecv{x_1}{z}$. 
%\tmcmt{Does it matter that we seem to switch freely between a row vector (above) and a column vector (here) for x?}
$\true$ corresponds to $\top$ and $\false$ to $\bot$.

%\tmcmt{Sudden introduction of this logical perspective is confusing. Get rid of it?  Also new fonts are suddenly used here for true and false.}
%
\Omit{
\begin{wrapfigure}{r}{4.5cm}
\vspace*{-7ex}
\scalebox{.65}{\import{figures/}{octagons.pspdftex}}
\caption{Example of an octagon}\label{octagon}
\vspace*{-5ex}
\end{wrapfigure} 
}
%
%In the interval domain $\intervals[\numvars]$,
For a program with $N=|\numvars|$ variables, the template 
matrix $\mat{C}$ for the interval domain $\intervals[\numvars]$, 
has $2N$ rows. Hence, it generates at most $2N$ inequalities, one
for the upper and lower bounds of each variable.
%
For octagons $\octagons[\numvars]$, we have at most $2N^2$
inequalities, one for the upper and lower bounds of each variable and
sums and differences for each pair of variables. 
%Fig.~\ref{octagon} gives an example of an octagon and its associated inequalities.
%
%A~fundamental difference between relational and 
Unlike a non-relational domain, a relational domain such as octagons 
requires the computation of a \emph{closure} in order to obtain a normal 
form, necessary for precise domain operation. 
The closure computes all implied domain constraints.  
%that is usually required for precise and efficient domain operations.  
An example of a closure computation for octagonal inequalities is
$\mathit{closure}((x-y \leq 4) \wedge (y-z \leq 5))=((x-y \leq 4) \wedge (y-z
\leq 5) \wedge (x-z \leq 9))$.
%
For octagons, closure is the most critical and expensive operator; it has  
cubic complexity in the number of program variables~\cite{pldi15}.  
We therefore compute closure lazily in template polyhedra domain in our abstract 
model search procedure. We refer the reader to Appendix~\ref{appendix:lazyclosure} 
for details of lazy closure computation. 
%\tmcmt{Will reference to an appendix be allowed in the final paper?}

\Omit{
In this paper, we use a product domain $\domain[\vars]=\booldomain^{|\boolvars|}\times\reldomain[\numvars]$ where
$\booldomain$ is the Boolean domain $\langle\{\true,\false,\bot,\top\},\Rightarrow,\wedge,\vee\rangle$ \tmcmt{Do we really need to introduce this notation here?}, 
$\boolvars$ are the Boolean variables in the program, 
and $\reldomain$ is a \emph{template polyhedra} domain over the 
numerical (bitvector) variables $\numvars$. 
}
%
%We instantiate $\reldomain$ with the template polyhedra domain~\cite{sriram}.
\Omit{
For notational convenience we will denote elements of
$\booldomain^{|\boolvars|}$ by their concretisations to propositional formulae.  
For example, the program given by Eq.~(\ref{eq:ssa}) has four Boolean variables
written as the vector $(g_0,g_1,c,g_2)$. Thus, we will denote an abstract 
value $(\top, \false, \true, \top) \in \booldomain^4$ as $\neg g_1 \wedge c$.
}

%-------------------------------------------------------------------------------
\subsubsection{Abstract Transformers}
%
%In a typical abstract interpretation based Galois-connection setting
%over an over-approximate domain, every concrete element has a unique
%over-approximate representation in the abstract.  Likewise, every
%concrete transformer is over-approximated by a unique abstract
%transformer.  We now define an abstract deduction transformer.
%
An abstract transformer $\abstrans{\domain}{\constraint}$ transforms an
abstract value $\absval$ through a constraint $\constraint$; it
\emph{deduces} information from $\absval$ and $\constraint$.  The best
transformer is
%
\begin{equation}\label{eq:abstrans}
\abstrans{\domain}{\constraint}(\absval)=\absval\meet\alpha(\{\val\mid \val\in\gamma(\absval), \val\models \constraint\})
\end{equation}
 where we write 
$\val\models \constraint$ if the concrete value $\val$ satisfies the constraint $\constraint$.
%$\forall \vec{y}\in c: (s[\vec{y}/\vec{\numvar}]
%\equiv \true)$, i.e., the constraint $s$ is valid when evaluated over
%all values $\vec{y}$ of its variables $\vec{\numvar}$ in the concretisation
%of $a$. 
Any abstract transformer that over-approximates the best abstract
transformer is a sound transformer and can be used in our algorithm.
%
For example, we can deduce $\abstrans{\domain}{x=2(y+z)}(\absval)=(0\leq y
\leq 2 \wedge\allowbreak 1 \leq y-z \leq 1\wedge\allowbreak -2\leq x\leq 6)$
for the abstract value $\absval=(0\leq y \leq 2 \wedge\allowbreak 1 \leq y-z
\leq 1)$.
%
%A detailed description of abstract transformer is presented in Section~\ref{sec:abst}.
We~denote the set of abstract transformers for a safety equation
$\formula$ using abstract domain $\domain$ by
$\abstransset=\{\abstrans{\domain}{\constraint}\mid
\constraint\in\constraints\}$.

%-------------------------------------------------------------------------------
\subsection{Properties of Abstract Domains}
%-------------------------------------------------------------------------------
%
An important property of a clause-learning SAT solver is that each
non-singleton element of the partial assignment domain can be 
decomposed into a set of \textit{precisely complementable} singleton
elements~\cite{dhk2013-popl}.  This property of domain elements
are necessary to learn elements that help to guide the model search away from the
conflicting region of the search space.  
% The complementation operator 
% in abstract domains is different from the notion of precise complements. 
%
Most numerical abstract domains, such as intervals and octagons lack
complements in general, i.e., not every element in the domain has a
precise complement.  However, these domain elements can be represented
as intersections of half-spaces, each of which admit a precise complement.
%\tmcmt{(does this mean 'lack complements in general' - i.e., NOT every element has a precise complement?}, but they \tmcmt{Grammar says that ``they'' here means the domains. Surely you mean the elements?} 
%
% We identify specific properties of domain elements necessary for 
% abstract model search and learning in abstractions.  
We formalise this in the sequel.
%
\begin{definition} 
A \emph{meet irreducible} $m$ in a complete lattice 
structure $A$ is an element with the following property.
\begin{equation}
\forall m_1, m_2 \in A: m_1 \meet m_2 = m \implies (m = m_1 \lor m = m_2), m \neq \top  
\end{equation}
\end{definition}
%
%Meet irreducibles in ACDCL correspond to the concept of literals in a SAT solver.
The meet irreducibles in the Boolean domain $\booldomain$ 
for a variable $x$ are $x$ and $\neg x$. The meet 
irreducibles in the template polyhedra domain are all elements 
that concretise to half-spaces, i.e., they can be represented 
by a single inequality. For the interval domain, these are 
$x \leq d$ or $x \geq d$ for constants $d$. 

%An important property of meet irreducibles in case of partial assignments 
%domain is that they have precise complements.  
%For example, the complement of $\{x \mapsto true \}$ is a 
%singleton element,$\{x \mapsto false \}$, in the partial assignments domain. 

\begin{definition}
A \emph{meet decomposition} $\decomp(\absval)$ of an abstract
element $\absval \in \domain$ is a set of meet irreducibles $M \subseteq \domain$ such that 
$\absval=\bigsqcap_{m\in M} m$.
%$forall m_i \in M, \meet(m_i) = a$, where $max(i) = |M|$.
\end{definition}

\noindent For polyhedra this intuitively means that each polyhedron can be
written as an intersection of half-spaces.
%
For example, the meet decomposition of the interval domain element
% $(x,y) \in [2,4]\times[3, 5]$ 
$decomp(2\leq x\leq 4 \wedge 3\leq y\leq 5)$ is
the set $\{x\geq 2, x\leq 4, y\geq 3, y\leq 5\}$.
% \langle x:[2, 4] \rangle \in ItvDom$ which gives set of meet irreducibles, 
% $\{ \langle y \succeq 3 \rangle, \langle y \preceq 5 \rangle, 
% \langle x \succeq 2 \rangle, \langle x \preceq 4 \rangle \}$, that are 
% precisely complementable.

\begin{definition} 
An element $\absval\in \domain$ is called \emph{precisely complementable}
iff there exists $\bar{\absval} \in \domain$ such that $\neg\gamma(\bar{\absval})=\gamma(\absval)$.
That is, there is an element whose negated concretisation equals
the concretisation of $\absval$.
%m\meet\bar{m}=\bot \wedge m\join\bar{m}=\top$.
%A complementable meet irreducible $\bar{M}$ of an abstract lattice $A$ is the complement of a meet 
%irreducible $M \in A$ such that $\bar{M} \in A$ and the concretisation of $M$ 
%is the set complement of $\bar{M}$.  
%If every meet irreducible in $A$ is complementable, then $A$ is said to
%be have complementable meet irreducibles.
\end{definition}

% An important property of a clause learning SAT solver 
% is that a partial assignments domain can be decomposed 
% into a set of precisely complementable singleton assignments.  
% This property of the partial assignments domain helps to 
% guide the model search away from the conflicting region 
% of the search space and allows case-based analysis.  Complementation 
% operator in abstract domains is different from the notion of precise
% complements. 
%
% \begin{definition}{(Precise complement)} Let $A$ be an lattice. A precise 
% complement of an abstract element $a \in A$ is an element $\bar{a} \in A$ 
% such that the negation of concretization $(\gamma)$ of $a$ is equivalent to 
% concretization of $\bar{a}$, that is, $\gamma(\bar{a}) = \neg \gamma(a))$.
% \end{definition}
%
The precise complementation property of a partial assignment lattice can
be generalised to other lattice structures. 
%Most numerical abstract domains, such as intervals, octagons, polyhedra, can be decomposed
%into half-spaces, which admits precise complements.  
%
For example, the precise complement of a meet irreducible $(x \leq 2)$ in
the interval domain over integers is $(x \geq 3)$, or the precise complement
of the meet irreducible $(x+y \leq 1)$ in the octagon domain over integers
is $(x+y \geq 2)$.  Our domain implementation supports precise
complementation operation.  However, standard abstract interpretation does
not require a complementation operator.  Hence, abstract domain libraries,
such as APRON~\cite{apron}, do not provide it.  But it can be implemented
with the help of a meet decomposition as explained above.
%
%\tmcmt{$\leftarrow$ This sentence does not make sense!}
\Omit{Numerical abstract domains that admit complementable decomposition
  are shown in Fig.~\ref{fig:complement}.}
% 
\Omit{
\begin{table}
\begin{center}
{
\begin{tabular}{l|l}
\hline
Domain & Precise Complements \\ \hline 
Interval & \((x \leq n) \quad \longrightarrow (x > n)\) \\ \hline
Octagon & \((x+y < 1) \quad \longrightarrow (-x-y < 0)\) \\ \hline
Equality & \((x==y) \quad \longrightarrow (x \neq y)\) \\ \hline 
\end{tabular}
}
\end{center}
\label{fig:complement}
\end{table}
}
%
\Omit {
Note that for many domains $A$, including template polyhedra,
most domain elements are not precisely complementable within $A$.
%
In fact, for template polyhedra all non-meet-irreducible elements $e$
(except $\bot$ and $\top$) are not precisely complementable,
whereas all meet irreducibles are precisely complementable.
%
Hence, we can complement each element in the meet decomposition of $e$ and
re-interpret the obtained set as a disjunction. 
%
%Thus the complement of the octagon in Fig.~\ref{octagon} 
%can be written as a disjunction of meet irreducibles:
%\[(x{\leq} -3) \lor (x{\geq} 2) \lor (y{\leq} -2) \lor (y{\geq} 3) \lor (x+y{\geq} 3) \lor (x-y{\geq} 2) \lor 
% (y-x{\geq} 4) \lor (-x-y{\geq} 3)\]
}

\Omit {
\begin{definition}{(Abstract Valuation)} An {\em abstract valuation} is a
mapping of variables to an element of abstract domain. For example, 
\Omit {
a mapping of variable $x$ to an interval environment is given by 
$\langle x \mapsto [2,5] \rangle$ or a 
}
a mapping of constraint variables $\{x,y\}$ to octagon environment 
is given by $\langle x-y \mapsto 0, y-x \mapsto 0 \rangle$.  
An abstract valuation is {\em atomic} if each variable is mapped to a singleton 
value or to $\bot$.  An abstract valuation $(v)$ abstractly satisfies a formula 
$\formula$ if for every variable $x$ in $\formula$, there is a concrete solution 
$(c)$ such that $c(x) \subseteq \gamma \circ v(x)$ holds. 
\end{definition}
}

%
%Another feature that abstract domain libraries do not provide is to
%track the reasons why certain deductions have been made. We come back
%to this point in Section~\ref{sec:abst}.

% The abstract transformer, $apost_{stmt}$, captures the effect of different program 
% statements in the abstract domain. The transformer is precise for octagonal 
% assignments $(x:=y+1)$ but imprecise for non-octagonal assignments $(x:=y+z)$, 
% as shown below.
% \[apost_{x:=y+1}(a) = b = \langle x-y \leq 1, y-x \leq 1 \rangle \qquad apost_{x:=y+z}(b) = \langle \top \rangle \]  

% \begin{definition}{(Abstract Deduction Transformer)} An abstract deduction
% transformer, $\widehat{ded_{\formula}}$ for a formula $\formula$ over an abstract 
% domain $A$ is a sound approximation of a concrete model transformer
% $ded_{\formula}$, given by $\widehat{ded_{\formula}} : A \rightarrow A$, such that 
% $\forall a \in A: \widehat{ded_{\formula}}(a) \in \{\top, \bot, m\}$, where 
% $m \in A$ is a meet irreducible.   
% \end{definition}

% Let us consider a formula $\formula = (x:=y-1)$ to be analyzed over 
% an interval abstract domain, $A = ItvDom$, and let $a = \langle y:[3, 5]
% \rangle \in ItvDom$, then $\widehat{ded_{\formula}}(a) = a \meet \langle x:[2, 4]
% \rangle$.  An abstract deduction transformer is typically computed in the form 
% of strongest post-condition or a weakest pre-condition of a formula in the 
% abstract domain.  

% For a program with $N$ variables, let $L$ be the total number of 
% meet irreducibles returned by a domain $D$.  For $D$ = {\em ItvDom}, the 
% maximum value of $L$ is $2*N$, whereas the maximum value of $L$ is 
% $2*(N^2)$ for $D$ = {\em OctDom}. Note that an octagon is the conjunction 
% of all octagonal inequalities in the set $L$.
%
%========================
%\input{literals-clauses}
%========================

% \Omit {
% The commonly used library for numerical abstract domains  
% is the APRON C library~\cite{apron}.  This library is 
% used for the static analysis of the numerical variables 
% of a program by abstract interpretation. APRON provides a 
% C API interface to various abstract domains and libraries 
% such as {\em BOX}, {\em OCTAGON}, {\em Convex Polyhedra} and
% {\em Linear Equalities} library.  The aim of such analysis is 
% to compute invariants over numerical variables in the 
% program~\cite{se2011}. 
% }
% \Omit {
% To this end, we implement our own template-based polyhedra domain and interval 
% domain which supports complementation operator.  
% %For example, the octagon in Fig.~\ref{octagon} can be written as a conjunction of:
% %\[(x>=-2) \land (x<=1) \land (y>=-1) \land (y<=2) \land (x+y<=2)
% %\land (x-y<=1) \land (y-x<=3) \land (-x-y<=2)\] 
% \pscmt{That's no valid motivation. We never complement a whole octagon, but just a
%  meet-irreducible. It's trivial to do that with APRON. The reason was
% a different one: APRON does not support all C operators, e.g. the bitwise
% operators.} 
% } 

\Omit{
A general polyhedral analysis is most expressive \pscmt{what does ``most'' mean?} but has exponential 
worst-case space and time complexity.  By contrast, template polyhedra 
are restricted since they can encode inequalities of the 
form $a_1x_1 + \ldots + a_nx_n \geq c$ \pscmt{it looks inconsistent to have $\geq$ here and $\leq$ a few lines below; also the coefficients here are $a$ whereas they are $c$ below; constant is $c$ here and $d$ below\ldots; also, $a$ is used for abstract value in later sections.}, where the coefficients 
$a_1, \ldots, a_n$ are fixed apriori.  Hence, the complexity of 
a template polyhedral analysis is in worst-case polynomial time in the 
size of the program and the domain~\cite{vmcai05}. 
}

\Omit{
Thus, the expressivity of template polyhedra lies between weakly relational domain 
such as intervals $(a \leq x_i \leq b)$, octagons $(\pm x_i \pm x_j \leq c)$, 
and strongly relational domain such as polyhedra domain.  
}

