\section{Abstract Conflict Driven Learning for Programs}
%
%
\begin{figure}[htbp]
\centering
\vspace*{-0.2cm}
\scalebox{.57}{\import{figures/}{acdlp-top.pspdftex}}
\caption{Architectural View of ACDLP \label{acdlp-top}}
\end{figure}
%
\begin{algorithm2e}[t]
\DontPrintSemicolon
\SetKw{return}{return}
\SetKwRepeat{Do}{do}{while}
%\SetKwFunction{assume}{assume}
%\SetKwFunction{isf}{isFeasible}
\SetKwData{conflict}{conflict}
\SetKwData{safe}{safe}
\SetKwData{sat}{sat}
\SetKwData{unsafe}{unsafe}
\SetKwData{unknown}{unknown}
\SetKwData{true}{true}
\SetKwInOut{Input}{input}
\SetKwInOut{Output}{output}
\SetKwFor{Loop}{Loop}{}{}
\SetKw{KwNot}{not}
\begin{small}
\Input{A program in the form of a set of abstract transformers $\abstransset$.}
\Output{The status \safe or \unsafe. %and a counterexample if \unsafe.
}
$\trail \leftarrow \langle\rangle$, $\reasons \leftarrow []$ \;
$\mathit{result} \leftarrow \deduce_{\propheur}(\abstransset,\trail,\reasons)$ \;
\lIf{$\mathit{result}$ = \conflict} {
  \return \safe}
\While{$true$} 
{
\lIf{$\mathit{result}$ = \sat} {
  \return \unsafe}
  $\decisionvar \leftarrow \decide_{\decheur}(\abs(\trail))$ \;
  $\trail \leftarrow \trail \cdot \decisionvar$ \; 
  $\reasons[|\trail|] \leftarrow \top$ \;
  $\mathit{result} \leftarrow \deduce_{\propheur}(\abstransset,\trail,\reasons)$\;
  \Do{$\mathit{result} = \conflict$} {
    \lIf{$\neg \analyzeconflict_{\confheur}(\abstransset,\trail,\reasons)$} {
      \return \safe
    }
    $\mathit{result} \leftarrow \deduce_{\propheur}(\abstransset,\trail,\reasons)$ \;
  }
}
\end{small}
\caption{Abstract Conflict Driven Learning $ACDLP_{\propheur,\decheur,\confheur}(\abstransset)$ \label{Alg:acdcl}}
\end{algorithm2e}
%
Figure~\ref{acdlp-top} present our framework called \emph{Abstract Conflict 
Driven Learning for Programs} that uses abstract model search and abstract 
conflict analysis procedures for safety verification of C programs.  The model
search procedure operates on an over-approximate domain of \rmcmt{program
traces} through repeated application of abstract deduction transformer, 
$\abstransel{}$, and decisions in order to search for a counterexample trace.  
If the model search finds a satisfying assignment
(corresponding deduction transformer is $\gamma$-complete), then ACDLP 
terminates with a counterexample trace, and the program is \emph{unsafe}.  
Else, if a conflict is encountered, then
it implies that the corresponding program trace is safe.  ACDLP then moves to the conflict 
analysis phase where it learns the reason for the conflict from partial safety proof using an
abstract abductive transformer, $\abd{}{}$, followed by a heuristic choice of conflict 
reason.  Similar to SAT solver, ACDLP chooses one conflict reason from multiple
incomparable reasons for conflict, so it operates over an under-approximate domain
of \rmcmt{conflict reason}.  The conflict analysis returns a learnt transformer and 
the model search is repeated with this new transformer.  Else,
if no further backtracking is possible, then ACDLP terminates and returns
\emph{safe}.  We describe the working of ACDLP in details in subsequent
sections.

The input to ACDLP (Algorithm~\ref{Alg:acdcl}) is a
program in the form of a set of abstract transformers
$\abstransset=\{\abstrans{\domain}{\sigma}|\sigma\in\Sigma\}$
w.r.t.\ an abstract domain~$\domain$.  Recall that the safety 
formula $\bigwedge_{\constraint\in\constraints} \constraint$ 
is unsatisfiable if and only if the program is safe.  
The algorithm is parametrised by heuristics for propagation $(\propheur)$, 
decisions $(\decheur)$, and conflict analysis $(\confheur)$.
%\rmcmt{Approximation of the concrete transformers in 
%$\abstransset$ are typically available in abstract domain in the 
%form of strongest-post condition or weakest pre-condition. } 
The algorithm maintains a propagation trail $\trail$ and 
a reason trail~$\reasons$.
%The propagation trail is initialized with an empty sequence.  
The propagation trail stores all meet irreducibles inferred by 
the abstract model search phase (deductions and decisions).  
The reason trail maps the elements of the propagation trail to the
transformers $\abstransel{}\in\abstransset$ that were used to
derive them. 
%
\begin{definition} 
The \emph{abstract value} $\abs(\trail)$ corresponding to 
the propagation trail $\trail$ is the conjunction of the 
meet irreducibles on the trail:
$\abs(\trail)=\bigsqcap_{m \in \trail}m$ with
$\abs(\trail)=\top$ if $\trail$ is the empty sequence.
\end{definition}
%
The algorithm begins with an empty $\trail$, a empty $\reasons$, and the
abstract value $\top$.  The procedure $\deduce$ (details in
Section~\ref{sec:deduce}) computes a greatest fixed-point over the
transformers in $\abstransset$ that refines the abstract value,
similar to the Boolean Constraint Propagation
step in SAT solvers.  If the result of $\deduce$
is \textsf{conflict} ($\bot$), the algorithm terminates with
\textsf{safe}.  Otherwise, the analysis enters into the while loop at line 4
and makes a new decision by a call to $\decide$ (see
Section~\ref{sec:decide}), which returns a new meet irreducible
$\decisionvar$.
%
%\pscmt{what happens if we cannot make any decision any more?
%  -- then we should either have deduced $\bot$ or $\abs(\trail)$ must
%  be $\gamma$-complete} 
%
We concatenate $\decisionvar$ to the trail~$\trail$.  The decision
$\decisionvar$ refines the current abstract value $\abs(\trail)$ represented
by the trail, i.e., $\abs(\trail\cdot\decisionvar)\sqsubseteq \abs(\trail)$.
%
% WE ExPLAIN THAT LATER
%However, $\decisionvar$ must be consistent with the state of the
%solver, i.e., $\abs(\trail\cdot \decisionvar)\neq \bot$.
%
For example, a decision in the interval domain restricts the range of 
intervals for variables.
%
We set the corresponding entry in the reason trail~$\reasons$ to $\top$
to mark it as a decision.  Here, the index of $\reasons$ is the size 
of trail $\trail$, denoted by $|\trail|$.
%
The procedure $\deduce$ is called next to infer new meet irreducibles
based on the current decision.  The model search phase
alternates between the decision and deduction until $\deduce$ returns
either \textsf{sat} or \textsf{conflict}.  

If  $\deduce$ returns  \textsf{sat}, then we have found an abstract value 
that represents models of the safety formula, which are counterexamples 
to the required safety property, and so ACDLP return \textsf{unsafe}.
%
% has found an abstract value that represents a set of models of $\formula$ or the
% deduction leads to a \textsf{conflict}.  Checking whether the abstract
% model concretises to a set of models of $\formula$; this
% corresponds to a $\gamma$-completeness~\cite{dhk2013-popl} check in
% abstract interpretation.  If $v$ is $\gamma$-complete, then it cannot
% be refined further.  Thus, the algorithm returns the abstract model
% $v$, which is a set of concrete models, and terminates with
% \textsf{unsafe} or the current abstraction is insufficient to
% determine the satisfiability of $\formula$.  
%
If  $\deduce$ returns  \textsf{conflict}, 
the algorithm enters in the $\analyzeconflict$ 
phase (see Section~\ref{sec:conflict}) to learn the reason for the conflict.   There can be multiple
incomparable reasons for conflict.
% \pscmt{do you have to explain this here?} -- based on the choice of Unique
%Implication Point (UIP)~\cite{cdcl}.  
ACDLP heuristically chooses one reason~$\conflictset$ and learns it 
by adding it as an abstract transformer to $\abstransset$. The analysis 
backtracks by removing the content of $\trail$ up to a point where it does not 
conflict with $\conflictset$.  ACDLP then performs deductions with the learnt 
transformer.  If $\analyzeconflict$ returns $\false$, then no further
backtracking is possible.  Thus, the safety formula is unsatisfiable
and ACDLP returns \textsf{safe}.

% A learnt clause must include asserting cuts which guarantees
% derivation of new meet irreducibles after backtracking. The clause
% learning and backtracking continues as long as the result of deduction
% is conflicting ($\bot$), that is, the abstract value does not
% abstractly satisfy the formula \pscmt{define}.  If no further
% backtrack is possible, then the algorithm terminates and $\formula$ is
% \textsf{safe}. Else, the algorithm makes a new decision and the above
% process is repeated until $\deduce$ returns \textsf{sat} or the
% algorithm backtracks to decision level 0 after a conflict in which
% case it returns \textsf{safe}.

%\paragraph{Solver State.}  \pscmt{This paragraph seems redundant. Doesn't this redefine $\trail$?} The state of a ACDLP solver is a tuple 
%of the form $\langle \mathcal{E}, S \rangle$.  Here, $\mathcal{E}$ 
%is a sequence of labelled information of the form $(m,s)$ where 
%$m$ is a meet irreducible and $s = \mathsf{decision}$ if $m$ is a decision, 
%or $s = \mathsf{deduction}$ if $m$ is inferred by a deduction. And $S$ is 
%a set of abstract deduction transformer \pscmt{???}.  A trail in a SAT solver 
%stores variable assignments of the form $(p, t)$, where $p$ is 
%a propositional variable that appears at most once in the trail 
%and $t$ is a truth assignment (true or false).  Whereas, a trail 
%$\trail$ in ACDLP contains a sequence of meet irreducibles 
%inferred by deduction or decision phase where a variable in trail 
%can be assigned \pscmt{constrained?} multiple times, each time with increasingly precise 
%bounds \pscmt{that's specific for a particular domain}.  
%\rmcmt{define trail refinement}
%\Omit {
%A current valuation $v$ is a meet of all elements of trail
%$\trail$, such that $v = \top$ when $\trail$ is 
%empty or $v=\underset{i \geq 0 \wedge i \leq |\trail|}{\meet m_i}$, where 
%$m_i \in \trail$.  A solver is in conflict if some clauses in
%$\abstransset$ is not abstractly satisfied by $v$.
%}
%\pscmt{define ``consistent with trail''}
