\subsection{Decisions}\label{sec:decide}
%
A decision $\decisionvar$ is a meet irreducible that refines the
current abstract value $\abs(\trail)$, when the result of the fixed-point 
computation through deduction is neither a {\em conflict} nor a {\em 
satisfiable model} 
of $\formula$.  A decision must always be consistent 
with respect to the trail $\trail$, 
i.e., $\abs(\trail\cdot \decisionvar)\neq \bot$.  A new 
decision increases the decision level by one. Given the 
current abstract value $\abs(\trail)$, the procedure $\decide$ 
in Algorithm~\ref{Alg:acdcl} heuristically returns a meet irreducible.


%Note that meet irreducibles in
%relational domains may involve several variables.
%
For example, a decision in the interval domain can be of the form 
$x R d$ where $R \in \{\leq,\allowbreak\geq\}$, and $d$ 
is the bound.  A decision in the octagon domain can specify relations 
between variables, and can be of the form $ax - by \leq d$, where 
$x$ and $y$ are variables, $a,b \in \{-1,0,1\}$ are coefficients, 
and $d$ is a constant.  The detailed description of the different 
decision heuristics in ACDLP is available at~\cite{extended}.
%\url{http://www.cprover.org/acdcl/}.

%\in \mathbb{R}\cup\{\infty\}$ is the bound of the inequality \pscmt{we have bit-vectors!}. 
%
%Hence, a decision can specify a relation between variables, for example 
%$(x \leq y)$.  

\Omit{
We call a meet irreducible that does not represent a valid decision a
\emph{singleton} meet irreducible.  
This is similar to a literal in a SAT solver that is already 
assigned $\true$ or $\false$, and thus cannot participate in 
a decision.
%
For template polyhedra, singletons are the meet irreducibles
corresponding to a pair of rows $\vec{c}_1,\vec{c}_2$ in matrix
$\mat{C}$ with $\vec{c}_1\vec{\numvar}=-\vec{c}_2\vec{\numvar}$
(i.e.\ \emph{matching} rows) and the corresponding bounds
$\numabsval_1=-\numabsval_2$.
%
A singleton in the interval domain corresponds to a singleton interval
such as $x\leq 1 \wedge -x\leq -1$.  For the octagon domain, $1 \leq x-y
\leq 1$ is a singleton.
%
%We cannot make decisions for template rows of singletons because there
%is no choice left. 
%
Note that for relational domains singletons do not necessarily
concretise to singleton sets of concrete values for the variables
involved.
}
%
%over a set of branching
%variables $\{B\} \subseteq \mathcal{V}$, a bound $c$, and a polarity
%($\leq$ or $\geq$).  For interval domain, $|B|=1$ since an interval
%meet irreducible is defined over a single variable.  For octagons,
%$|B|=2$.  A decision adds a new meet irreducible $m$ to the trail.
%Subsequently, the new state of the solver is defined over $m$ and
%corresponding label information $s=decision$, which is described
%below.
%\[decide: \quad (\mathcal{E},S) \rightarrow (\mathcal{E}(m,s),S) \]

%Let $\mathcal{V}$ be a set of all singletons and non-singletons.  
%Otherwise, the variable is non-singleton 
\Omit {
We have implemented several decision heuristics in ACDLP: {\em ordered}, 
{\em longest-range}, {\em random}, and the {\em Berkmin}~\cite{eugoldberg07} 
decision heuristic.  The {\em ordered} decision heuristic 
%creates an ordering among meet irreducibles, 
makes decisions on meet irreducibles that involve conditional 
variables (variables that appear in conditional branches) first 
before choosing meet irreducibles with numerical variables.  
%The ordered heuristic gives an effect of trace partitioning~\cite{toplas07}.
%
The {\em longest-range} heuristic simply keeps track of the bounds
$\numabsval_l,\numabsval_u$ of matching template rows, which are 
%\footnote{These are template rows with row vectors $\vec{c}$, $\vec{c}'$ such that $\vec{c}=-\vec{c}'$.}
row vectors $\vec{c}$, $\vec{c}'$ such that $\vec{c}=-\vec{c}'$.
%\pscmt{[that has become a bit hard to understand since some of the definitions have been removed]} 
$\numabsval_l\leq \vec{c}\vec{x}\leq \numabsval_u$, picks the one with the longest range
$\numabsval_u-\numabsval_l$, and randomly returns the meet irreducible
$\vec{c}\vec{x}\leq
\lfloor\frac{\numabsval_l+\numabsval_u}{2}\rfloor$ or its
complement. This ensures a fairness policy in selecting a variable
since it guarantees that the intervals of meet irreducibles are
uniformly restricted.
%
The {\em random} decision heuristic arbitrarily picks a meet irreducible  
for making decision. 
%
%The {\em relational} decision heuristics is only relevant for relational 
%abstract domains.  
%
The {\em Berkmin} decision heuristic is inspired by the 
decision heuristic used in the Berkmin~\cite{eugoldberg07} SAT solver.  
The Berkmin heuristic %is currently implemented for interval constraints only.  
%The heuristic 
keeps track of the activity of %an interval 
meet irreducibles that participate in conflict clauses. 
Based on the most active meet irreducible, ranges are split 
similar to the {\em longest-range} heuristic.
}

\Omit {
as well as variables that actively contribute to conflicts but do not explicitly 
appear in conflict clauses.  The set of conflict clauses is 
organised chronologically with the top clause 
as the one deduced in the last.  A branching variable is chosen among the 
free variables whose literals are in the top unsatisfied conflict clause.  
A similar decision heuristic is also implemented in Chaff~\cite{chaff} SAT 
solver, that computes the activity of a variable as the number of occurrences 
of that variable in conflict clauses only. 
}
%
%A bound of a meet irreducible is heuristically chosen to be an
%approximation of the arithmetic average of the current bounds.
%However, the polarity ($\leq$ or $\geq$) of a meet irreducible is
%chosen randomly.
