\section{Abstract Model Search for Template Polyhedra}\label{sec:deduce}

\Omit{
\begin{figure}[t]
\scriptsize
\begin{tabular}{l|l|l}
\hline
C program & SSA & Octagon \\
\hline
\begin{lstlisting}[mathescape=true,language=C]
int main() {
 unsigned x, y;
 __CPROVER_assume(x==y);
 x++;
 assert(x==y+1);
}
\end{lstlisting}
&
\begin{minipage}{4.40cm}
$\begin{array}{l@{\,\,}c@{\,\,}l}
SSA &\iff& ((g0 == TRUE) \land \\
    &    & (cond == (x == y)) \land \\
    &    & (g1 == (cond \&\& guard0)) \land \\
    &    & (x' == 1u + x) \land \\
    &    & (x' == 1u + y || !1))
\end{array}$
\end{minipage}
&
\begin{minipage}{3.75cm}
$\begin{array}{l@{\,\,}c@{\,\,}l}
C &\iff& ((x' > 1) \land (-x'-y < -2) \land \\
  &    & (-x-x' < -2) \land (y-x' < 0) \land \\                                                                
  &    & (x-x' < 0) \land (y > 0) \land \\
  &    & (x > 0) \land (-x'-y < 0) \land \\
  &    & (x+y > 1) \land (y-x < 1) \land \\
  &    & (x'-y < 2) \land (x-y < 1) \land \\
  &    & (x+y > 0) \land (x+x' > 0) \land \\
  &    & (x'-x < 2))
\end{array}$
\end{minipage}
\\
\hline
\end{tabular}
\caption{C Program, corresponding SSA and Octagonal Inequalities}
\label{ssa}
\end{figure}
}

\begin{algorithm2e}[t]
\DontPrintSemicolon
\SetKw{return}{return}
\SetKwData{sat}{sat}
\SetKwData{conflict}{conflict}
\SetKwData{unsat}{unsat}
\SetKwData{unknown}{unknown}
\SetKwData{true}{true}
\SetKwInOut{Input}{input}
\SetKwInOut{Output}{output}
\SetKwFor{Loop}{Loop}{}{}
\SetKw{KwNot}{not}
\begin{small}
\Input{A program in the form of a set of abstract transformers $\abstransset$,
a propagation trail $\trail$, and a reason trail $\reasons$.}
\Output{\sat or \conflict or \unknown}
$\worklist \leftarrow \initworklist_{\propheur}(\abstransset)$ \;
\While{$!\mathit{worklist.empty}()$} 
{
  $\abstransel{\subdomain} \leftarrow \mathit{worklist.pop}()$ \; 
  $\absval \leftarrow \abstransel{\subdomain}(\abs(\trail))$\;
  \uIf{$\absval = \bot$} {
    $\reasons[\bot] \leftarrow \abstransel{\subdomain}$ \;
    $\mathit{worklist.clear}()$ \;
    \return \conflict \;
  }
  \uElse
  {
    $\newdeductions=\onlynew(\absval)$\;
%    $\newdeductions=\decomp(\absval)\setminus\decomp(\abs(\trail))$\;
    $\trail \leftarrow \trail \cdot \decomp(\newdeductions)$ \; 
    $\reasons[|\trail|] \leftarrow \abstransel{\subdomain}$ \;
    $\updateworklist_{\propheur}(\worklist, \newdeductions, \abstransel{\subdomain},  \abstransset)$ \; 
  }
}
\lIf{$\abstransset$ is $\gamma$-complete at $\abs(\trail)$} {
  \return \sat
}
 \return \unknown \;

\end{small}
\caption{Abstract Model Search $\mathit{deduce}_{\propheur}(\abstransset,\trail,\reasons)$ \label{Alg:ms}}
\end{algorithm2e}

Model search in a SAT solver has two steps: {\em deductions}, which are
repeated application of the unit rule (also called Boolean Constraint
Propagation, or BCP), to refine current partial assignments, and {\em
decisions} to heuristically guess a value for an unassigned literal.  BCP
can be seen to compute greatest fixed point over the partial assignment
domain~\cite{dhk2013-popl}.  Below, we present an abstract model search
procedure that computes a greatest fixed point over abstract
transformers~$\abstrans{\domain}{\constraint}$.
 
%The unit rule overapproximates a model transformer and deduction 
%computes a greatest fixed point over the partial assignments
%domain~\cite{dhk2013-popl}.  We present an abstract model search procedure 
%that computes a greatest fixed point over meet irreducible deduction 
%transformer in $S$.  
%

%-------------------------------------------------------------------------------
\subsection{Parametrised Abstract Transformers} \label{sec:abst}
%-------------------------------------------------------------------------------
\Omit{
To make our algorithm efficient, we have to focus the abstract
transformers on performing only the minimally necessary
work. 
}

The key considerations for an abstract transformer are precision and
efficiency.  A~precise transformer is usually less efficient than a more
imprecise one.  In this paper, we present a specialised variant of the
abstract transformer to compute deductions called \emph{Abstract Deduction
Transformer} (ADT), which is parametrised by a given \emph{subdomain}
$\subdomain\subseteq \domain$.
%
A subdomain contains a chosen subset of the elements in $\domain$ including
$\bot$ and $\top$ that forms a lattice.
%
The use of a subdomain serves two purposes -- 
%\begin{compactenum}[(1)]
%\item 
  a) It allows us elegantly and flexibly to guide the deductions in 
%\emph{propagation} heuristics to guide deductions, e.g. 
  {\em forward}, {\em backward} or {\em multi-way} direction, which 
  in turn affects the analysis precision, and 
%\item
  b) It makes deductions more efficient, for example by performing lazy closure
  in template polyhedra domain. For space reasons, we refer the reader to 
  Appendix~\ref{appendix:lazyclosure} for details of the lazy closure operation. 
%\end{compactenum}
%
An ADT is defined formally  as follows. 
\begin{equation}\label{eq:at2}
\abstrans{\domain}{\constraint}^\subdomain(\absval)=\absval\meet_\domain \alpha_\subdomain(\{\val\mid \val\in\gamma_\domain(\absval), \val\models \constraint\})
\end{equation}
For $\subdomain=\domain$, the ADT is
identical to the abstract transformer defined in
Eq.~(\ref{eq:abstrans}) in Section~\ref{sec:domains}.  Note 
that a restricted subdomain makes a transformer less 
precise but more efficient.  Conversely, an
unrestricted subdomain make a transformer more precise, but less
efficient. Therefore, we have the property
$\abstrans{\domain}{\constraint}^\domain(\absval)\sqsubseteq
\abstrans{\domain}{\constraint}^\subdomain(\absval)$.
%
\Omit{Thus, the
parametrisation helps us fine-tune the precision and efficiency of
deductions.}
\Omit{Furthermore, the choice of subdomain internally
  guides the deduction in {\em forward}, {\em backward} or {\em
    multi-way} direction which are described next.}
To illustrate point (1), 
we give examples that demonstrate how the choice of
subdomain influences the propagation direction:

\paragraph{Forward Transformer.} 
%
For an abstract value
$\absval=(0\leq y \leq 1 \wedge 5\leq z)$, $\constraint=(x=y+z)$, 
and $L={\intervals[\{x\}]}$,  we have
$\abstrans{\intervals[\{x,y,z\}]}{x=y+z}^{\intervals[\{x\}]}(\absval)=\absval\meet(x\geq
6)$.
Assuming that the equality $x=y+z$ originated from an assignment to $x$,
this performs a right-hand side (rhs) to left-hand side (lhs) propagation and
hence emulates a forward analysis. 

\paragraph{Backward Transformer.} 
For an abstract value $\absval=(0\leq x \leq 10 \wedge 0\leq y \leq 1 \wedge 5\leq z)$, 
$\constraint=(x=y+z)$, and $L={\intervals[\{y,z\}]}$, we have
$\abstrans{\intervals[\{x,y,z\}]}{x=y+z}^{\intervals[\{y,z\}]}=\absval\meet (z\leq 10)$. 
This performs an lhs-to-rhs propagation and hence emulates a backward analysis.  

\paragraph{Multi-way Transformer.} 
%
\Omit{We call the propagation with arbitrary subdomains,e.g. 
$\subdomain=\domain$, \emph{multi-way propagation}, which is able
to simultaneously perform forward and backward propagation.}
For an abstract value $\absval=(c\leq1 \wedge c\geq1 \wedge x\leq5 \wedge x\geq5)$, 
$\constraint=((c = (x=y)) \wedge y=y+1)$ and $L={\intervals[\{c,x,y\}]}$, we have 
$\abstrans{\intervals[\{c,x,y\}]}{\constraint}^{\intervals[\{c,x,y\}]}=\absval\meet(y\leq6
\wedge y\geq6)$.  This performs an lhs-to-rhs propagation for $c=(x=y)$ and rhs to lhs propagation
for $y=y+1$ and hence emulates a multi-way analysis.  
%
% $\abstrans{\intervals[\{c,x,y\}]}{y=y+1}^{\intervals[\{c,x,y\}]}=\absval\meet(y\leq6 \wedge y\leq6)$.  
% We apply the ADTs in sequence for the
% constraints $\constraint=(c == (x==y))$ and $\constraint=(y=y+1)$ with the
% setting $L=D$ for all $\constraint$.

\Omit{
% has already been said above
For efficiency reasons, an optimisation facilitated by subdomains is 
\emph{lazy closure} computation for deductions involving relational 
constraints in template polyhedra domain. For space reasons, we omit 
the description here and refer the reader to Appendix~\ref{appendix:lazyclosure}.}
%
\Omit{
Note that a restriction to a subdomain makes a transformer less
precise.
}

\Omit{
\paragraph {\textbf{Lazy closure computation}} 
Computing closure for relational domains, such as octagons is an 
extremely expensive operation~\cite{pldi15}.  An advantage of our 
formalism in Eq.~(\ref{eq:at2}) is that
%we can compute the
the \emph{closure} operation for relational domains can be computed 
in a lazy manner. To this end, we construct a subdomain 
$\subdomain=\makesubdomain_\domain(\subvars)$ for $\abstransel{\subdomain}$,
which is sufficient to perform one step of the closure operation when 
$\abstransel{\subdomain}$ is applied.
%
For example, let us consider $\domain=\octagons[\{x,y,z\}]$ and
$\subvars=\{y\}$. An octagonal inequality relates at 
most two variables. Thus it is sufficient to consider the subdomain
$\makesubdomain_\domain(\{y\})=\octagons[\{y\}]\cup\octagons[\{x,y\}]\cup\octagons[\{y,z\}]$,
which will compute the one-step transitive relations of~$y$ with each
of the other variables. 
%
Only if a ADT subsequently makes new deductions 
on $x$ or $z$, then the next step of the closure will be computed through 
the subdomain $\octagons[\{x,z\}]$.
\Omit{
Hence, when the ADT is applied we do not
compute the full closure in the full domain,
but we compute only a single step of the closure in a restricted
domain, which makes single deduction steps more efficient.
}
Hence, an application of ADT does not 
compute the full closure in the full domain, but compute only a 
single step of the closure in a restricted domain, which makes each 
deduction step more efficient.  Thus, we delay the closure operation 
until the point where it is absolutely necessary.  This makes our deductions 
in relational domain more efficient.  
}
%
\Omit{
If $\abstransel{\subdomain}$ deduces new information about $x$ or $z$
then the next step of the closure will be computed by the worklist
mechanism of Algorithm~\ref{Alg:ms} that we describe next.
}
%
 
%If this indeed affects $x$ and~$z$ then their
%relations will be inferred in a later step of the propagation
%algorithm, thus gradually computing the closure. 

%Note that this might
%increase the number of propagation steps, but it reduces the size of
%the subdomain used in the ADT.

%Moreover, we want to know what the reasons for a specific deduction are.
%\[ded(s,a,L)=\{R\rightarrow d\mid R=reasons(s,a,L,d), d \in decomp(\llbracket s \rrbracket^\sharp(a,L))\}\]
%where 
%$reasons(s,a,L,d)=\text{argmin}_{R\in\{R'\mid R'\subseteq decomp(a), \llbracket s\rrbracket^\sharp(R,L)=d\}} |R| $.

%However, the choice of a subdomain makes the deductions more efficient.
%Thus, the choice of a subdomain helps to fine-tune the deductions 
%during the propagation phase.

%We construct the subdomain from a set of variables $\vars$ such that $\subdomain_\domain(\vars)$ is the set of meet irreducibles $\in \domain$ that contain at last one variable in $\vars$ and most one variable that is not in $\vars$. \pscmt{split into two parts}
%
%For example, $\subdomain=\domain[\{y\}]$ being the octagons domain over variables $\{x,y,z\}$ contains all octagonal meet irreducibles involving $\{y\}$, and the pairs $\{x,y\}$ and $\{y,z\}$, but not the meet irreducibles\pscmt{wrong} over $\{x\}$, $\{z\}$ and~$\{x,z\}$.

%The first advantage of this definition is that 
%The parameterisation of the ADT with a subdomain 
%$\subdomain$ allow us to guide the propagation in 
%{\em forward}, {\em backward} or {\em multi-way} direction.
%-------------------------------------------------------------------------------
\subsection{Algorithm for the Deduction Phase}
%-------------------------------------------------------------------------------
%
Algorithm~\ref{Alg:ms} presents the deduction phase $\deduce$ in 
our abstract model search procedure.  The input to $\deduce$ is 
the set of abstract transformers, a propagation trail ($\trail$) 
and a reason trail~($\reasons$).  Additionally, the procedure 
$\deduce$ is parametrised by a propagation heuristic ($\propheur$). 
We write the ADT 
$\abstrans{\domain}{\constraint}^\subdomain$
as $\abstransel{\subdomain}$ in Algorithm~\ref{Alg:ms}. 
%
The algorithm maintains a {\em worklist}, which is a queue that contains 
ADTs.  The propagation heuristics provides two 
functions $\initworklist$ and $\updateworklist$.
The order of the elements in the worklist and the subdomain $\subdomain$ 
associated with each ADT ($\abstransel{\subdomain}$) 
determine the propagation strategy (forward, backward, multi-way).
These two functions construct a subdomain ($\subdomain$)
for $\abstransel{\subdomain}$ 
by calling the function $\makesubdomain$ such that 
$\subdomain=\makesubdomain_\domain(\subvars)$, where $\subvars$ are 
the variables that appear in $\abstransel{\subdomain}$. 
\Omit{
The {\em forward} propagation strategy initialises the worklist with
ADTs that contain constants in the
right-hand side; the subdomain is constructed via $\makesubdomain_\domain$
from variables in the left-hand side of the equality constraints
originating from the assignment statements in the program.
%
The {\em backward} propagation strategy initialises the worklist 
with the assertions; the subdomain is constructed from the 
right-hand side variables.
%
The {\em multi-way} propagation strategy initialises the worklist 
with the set of all transformers; corresponding subdomains 
contains the variables occurring in the respective transformers.
}
%
The abstract value $\absval$ is updated upon the application of 
$\abstransel{\subdomain}$ in line~4 in Algorithm~\ref{Alg:ms}. 
The function
$\onlynew(\absval)=\bigsqcap(\decomp(\absval)\setminus\decomp(\abs(\trail)))$
is used to filter out all meet irreducibles that are already on the trail
in order to obtain only new deductions ($\newdeductions$) when applying 
the ADT (shown in line~10).
%
Depending on the propagation heuristics, $\updateworklist$ adds
ADTs $\abstransel{\subdomain}$ to the 
worklist that contain variables that appear in $\newdeductions$, and 
updates the subdomains of the ADTs in the worklist 
to include the variables in $\newdeductions$ (shown in line~13).
%
%
\Omit{
If $\bot$ is deduced we return \textsf{conflict}.
Otherwise, when eventually a fixed point has been reached, i.e.\ the worklist is empty, then the abstract value $\abs(\trail)$ is checked whether it is 
$\gamma$-complete~\cite{dhk2013-popl}. 
%
It is $\gamma$-complete if all concrete values in $\gamma(\abs(\trail))$ satisfy $\formula$.
Otherwise, the algorithm returns \textsf{unknown} and ACDLP makes a new decision.}

If $\abstransel{\subdomain}$ deduces $\bot$, then 
the procedure $\mathit{deduce}$ returns \textsf{conflict} (shown in line~8).
Otherwise, when a fixed-point is reached, i.e.~the worklist is empty, we check whether
the abstract transformers $\abstransset$ are $\gamma$-complete~\cite{dhk2013-popl} for the current abstract value $\abs(\trail)$ 
(shown in line~15).
%
Intuitively, this checks whether all concrete values in 
$\gamma(\abs(\trail))$ satisfy the safety formula $\formula$, where 
$\formula:= \bigwedge_{\constraint\in\constraints} \constraint$ is obtained 
from the program transformation (as defined in Section~\ref{sec:program}).
%
If it is indeed 
$\gamma$-complete, then $\mathit{deduce}$ returns \textsf{sat}.  Otherwise, the 
algorithm returns \textsf{unknown} and ACDLP makes a new decision.    

