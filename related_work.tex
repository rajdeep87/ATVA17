\section{Related Work}
%
The work of~\cite{franzle} presents a tight integration of SAT solving
with interval based arithmetic constraint solving to handle large constraint
systems. 
%
Silva et. al.~\cite{sas12} present an abstract interpretation account of 
satisfiability algorithms dervied from DPLL procedures.  
%
The work of~\cite{tacas12} is a very early instantiation of abstract 
CDCL~\cite{sas12} as an interval-based decision procedure for programs, 
but in a purely logical settings.  
%
A similar technique that lifts DPLL(T) to programs is Satisfiability Modulo 
Path Programs (SMPP)~\cite{SMPP}. SMPP enumerates program paths using a SAT 
formula, which are then verified using abstract interpretation.  
%
The work of \cite{DBLP:conf/esop/MineBR16} proposes an algorithm inspired by 
constraint solvers for inferring disjunctive invariants using intervals.
%
The lifting of CDCL to first-order theories is proposed in~\cite{dpll,ndsmt}.
%
\Omit{ operates on a fixed first-order partial assignment lattice
  structure, where first-order variables are mapped to domain values,
  similar to constants lattice in program analysis.  } 
  Unlike previous work that operates on a fixed first-order lattice, however 
  ACDLP can be instantiated with different abstract domains.  This
  involves model search and learning in abstract lattices. ACDLP is not, however,
  similar to abstraction refinement. ACDLP works on a fixed
  abstraction. Also, transformer learning in ACDLP does not soundly over-approximate
  the existing program transformers. Hence, transformer learning in ACDLP is
  distinct from transformer refinement in classical CEGAR. 

\Omit { The
  abstract lattice in natural domain SMT does not have complementable
  meet irreducibles, and therefore does not support generalized clause
  learning~\cite{sas}.  On the other hand, most abstract lattice have
  complementation property, thus enabling ACDCL to perform generalized
  clause learning.  } 
%
%\cite{DBLP:journals/fmsd/BrainDGHK14} presents a similar decision and learning
%based approach for heap-manipulating programs, but uses a simplistic conflict analysis.
%
%\cite{DBLP:conf/vmcai/PelleauMTB13} describes a constraint solving algorithm
%using relational numerical domains, but performs model search only.
%
%
% is presented in . They
%use a relational domain, but focus on the model search, whereas the
%conflict analysis is simplistic and restarts the algorithm with the
%learnt clause instead of backtracking.
%}

