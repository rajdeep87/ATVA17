\section{Introduction}

%%%%%%%%%%%%%%%%%%%%%%% AI %%%%%%%%%%%%%%%%%%%%%%%%
%
Static program analysis with abstract interpretation has been widely 
used to verify properties of safety-critical systems~\cite{CC77}.  
Static analyses commonly aim to compute program invariants as 
fixed-points of abstract transformers.  Abstract states are chosen 
from a lattice that has meet $(\sqcap)$ and join $(\sqcup)$ operations; 
the meet precisely models set intersection (or conjunction, taking a 
logical view), and the join over-approximates set union (or disjunction).  
Over-approximation in the join operation is one of the sources of 
precision loss that can yield false alarms.  Typical abstract domains 
are non-distributive; suppose $a$ and $b$ together represent the
abstract semantics of a program and $c$ represents a set of abstract
behaviours that violate the specification.  In a non-distributive domain,
$(a \sqcup b) \sqcap c$ can be strictly less precise than $(a \sqcap c)
\sqcup (b \sqcap c)$.  This means that in typical abstract domains,
analysing program behaviours separately can improve the precision of the
analysis.  Usual means to address false alarms therefore include not only
the use of richer abstract domains, but also of refinements that delay joins
or perform some form of case-splitting.  Such techniques trade off higher
precision against lower efficiency and may be susceptible to case
enumeration behaviour.

%%%%%%%%%%%%%%%%%%%%%%% CDCL %%%%%%%%%%%%%%%%%%%%%
By contrast, model checking (MC)~\cite{mc-book} can be seen to operate on
distributive lattice structures that represent disjunction without loss
of precision.  Classical MC directly operates on distributive
representations, such as BDDs, while more recent implementations use SAT
solvers.  SAT solvers themselves operate on partial assignments, which are
non-distributive structures.  To handle disjunction, case-splitting is
performed~\cite{sas12}.  Propositional SAT solvers solve large formulae, and
are often able to avoid enumerating cases.  The impressive performance of
modern solvers is credited to well-tuned decision heuristics and
sophisticated clause learning algorithms.  Collectively, these algorithms
are referred to as \emph{Conflict Driven Clause Learning}
(CDCL)~\cite{cdcl}.  An obvious idea is to lift CDCL from the domain of 
partial assignments to other non-distributive domains.

%%%%%%%%%%%%%%%%%%%%%%% ACDCL %%%%%%%%%%%%%%%%%%%%%
Abstract Conflict Driven Clause Learning (ACDCL,~\cite{dhk2013-popl}) is one 
such lattice-based generalization of CDCL.  ACDCL is a general algorithmic 
framework, parameterized by a concrete domain $C$  and an abstract domain
$A$. Classical CDCL can be viewed as an instance of ACDCL 
in which $C$ is the set of propositional truth assignments and $A$ the 
domain of propositional partial assignments~\cite{leo-thesis}.  Since 
the concrete domain of interest is a parameter to the framework, 
ACDCL can in principle be used to build both 
\emph{logical decision procedures}~\cite{DBLP:journals/fmsd/BrainDGHK14} 
and \emph{program analyzers}.  In the former case, the concrete domain is the 
set of candidate models for the formula; in the latter case, it is the 
set of program traces that may lead to an error.  Haller et. al. 
in~\cite{DBLP:journals/fmsd/BrainDGHK14} illustrate the 
first idea by presenting an ACDCL based floating-point decision procedure.  
%

%%%%%%%%%%%%%%%%%%%%%%% ACDLP %%%%%%%%%%%%%%%%%%%%%

In this paper, we present an extension of ACDCL to program analysis.
We call our framework \emph{Abstract Conflict Driven Learning for Programs}
(ACDLP).   ACDLP uses decision and learning techniques to operate over
non-distributive lattices for safety checking of C programs.  The key insight
is to use decisions and learning to precisely reason about disjunctions in
non-distributive domains, thereby automatically refining the precision of
analysis.  We~introduce two central components of the framework: an abstract
model search algorithm for programs that searches for counterwitness and an 
abstract conflict analysis procedure that learns new transformer that
approximate set of unsafe traces.  
%%%%%%%%%%%%%%%%%%%%%%% Domain %%%%%%%%%%%%%%%%%%%%%
We illustrate the application of our framework to program analysis 
using a \textit{template polyhedra abstract domain}~\cite{tacas08}, which 
include most of the commonly used abstract domains, such as boxes, 
octagons, zones and TCMs.  

We present an experimental evaluation of our analyser compared 
to CBMC~\cite{cbmc.tacas:2004}, which uses propositional solvers, and to 
Astr{\'e}e~\cite{DBLP:conf/pldi/BlanchetCCFMMMR03}, a commercial abstract 
interpretation tool.  Our experiments suggest that ACDLP can be seen as a 
technique to improve efficiency of SAT-based BMC. Additionally, it can 
also be perceived as an automatic way to improve the precision of abstract 
interpretation for bounded unwindings of programs.  We make the following 
contributions.
%
\vspace{-3mm}
\begin{enumerate}
\item A novel program analysis algorithm that lifts model search and conflict
analysis procedures over a template polyhedra abstract domain. These techniques
are embodied in our tool, \emph{ACDLP}, for automatic bounded 
safety verification of C programs.
\item We present a parameterized abstract transformer for counterwitness 
  search that guides the analysis in forward, backward and multi-way direction.
\item A UIP-based transformer learning over template polyhedra abstract domain 
  through abductive reasoning for effective conflict analysis. 
\end{enumerate}
%

\Omit{
While the focus of this paper is the analysis of loop-free unwindings of programs, our technique is in 
principle compatible with abstract-interpretation-based handling of loops using 
widenings~\cite{leo-thesis}.  Although, finding non-trivial trace abstraction 
that satisfy properties of ACDCL is an open problem.  Finally, we present the 
results of experimental evaluation of our analyser compared to CBMC~\cite{cbmc.tacas:2004}, 
which uses propositional solvers, and to Astr{\'e}e~\cite{DBLP:conf/pldi/BlanchetCCFMMMR03}, 
a commercial abstract interpretation tool. 
}
%%%%%%%%%%%%%%%%%%%%%%% Experiments %%%%%%%%%%%%%%%%%%%%%
%\vspace{-7mm}

% This abstract domain captures many
% relationships between program variables that intervals cannot represent.
% TM - INTERVALS don't express relationships at all. 

%Fundamentally, the abstract domain of polyhedra~\cite{polyhedra} is a 
%complete relational domain that capture all linear inequalities 
%between program variables.  
%It is a `relational' domain, in the sense that
%its elements express relationships between two or more program variables. 
%By contrast, the interval domain expresses constraints on individual program variables.


\Omit {


%This enables an ACDCL-based analyser to automatically refine an imprecise 
%analysis and thus prevents enumeration behaviour.  

%To date, ACDCL has been instantiated as a decision procedure for the
%first-order theory of floating-point arithmetic~\cite{DBLP:journals/fmsd/BrainDGHK14}, 

%\rmcmt{state other limitations}.  
that is, given abstract domain elements $a$, $b$, $c$, 
$(a \sqcup b) \sqcap c \sqsupseteq (a \sqcap c) \sqcup (b \sqcap c)$.

We provide a theoretical recipe and a practical instantiation for 
generalising CDCL architecture to arbitrary abstract domains.   

\paragraph{Abstract Domain.} 
Limited expressivity of abstract domains may lead to imprecise 
over-approximation which gives rise to false alarms. Evidently, the 
verification of many programs requires more expressive numerical domains, 
such as polyhedra for analysis.  These domains capture numerical 
relationship between program variables that intervals cannot express.  
To this end, we provide a theoretical recipe and a practical instantiation 
for generalising CDCL architecture to arbitrary abstract domains, thus 
lifting it to richer lattice structures. 

The key motivation of our work is to combine the precision of a SAT solver 
and the efficiency of an abstract interpreter to present a new class of 
program analysers.  

\rmcmt{Introduce ACDLP here with a distinct notion for programs}
Whereas the propositional reasoning of SAT solvers can be lifted easily from the 
Boolean lattice to other non-relational domains, application to relational 
domains is challenging because of the relational properties of inferred 
deductions and the complexity of the closure operation in fixed-point computation 
during the propagation phase.  In this paper, we identify specific modifications 
to the CDCL algorithm that are necessary to lift propositional CDCL to 
template polyhedra domain. 

\rmcmt{Present program and property driven trace partitioning (with inline example with octagons)} 
}
