\section{Abstract Conflict Analysis in Template Polyhedra Domain}\label{sec:conflict}
%
Propositional conflict analysis with FIRST-UIP~\cite{cdcl} can be seen 
as abductive reasoning that under-approximates a 
set of models that do not satisfy a formula~\cite{sas12,dhk2013-popl}.  
Below, we present an abstract conflict analysis procedure, $\analyzeconflict$ of 
Algorithm~\ref{Alg:acdcl}, that uses a 
domain-specific abductive transformer for effective learning. 
A conflict analysis procedure involves two steps: {\em abduction} 
and {\em heuristic choice for generalisation}. Abduction infers 
possible generalisations for a conflict which is followed by 
heuristically selecting a generalisation.  The main idea of abductive reasoning is to 
iteratively replace a singleton assignment $s$ in the conflict 
reason by a partial assignment that is sufficient to infer $s$.  
Informally, an abstract abductive transformer, $\abd{}{}(A)$, 
for a given formula $\formula$ adds abstract models to the input set
$\allval$, that do not satisfy $\formula$. 
%
For example, an abstract abductive transformer for $\formula=\{x=y+1\}$ 
is given by, $\abd{x=y+1}{\intervals}(x \geq 0)=(x \geq 0 \cup y \leq -2)$.
%
Conflict abduction is performed by obtaining cuts through markings in the 
trail $\trail$, by the application of abstract Unique Implication Point (UIP) 
search algorithm~\cite{cdcl}.  Every cut in the graph is a reason for
conflict.\\ 
%
\begin{figure}[t]
\centering
\scalebox{.55}{\import{figures/}{oct_implgraph.pspdftex}}
\caption{\label{uip} Finding Abstract UIP in Octagon Domain}
\end{figure}  
%


\noindent \textbf{Abstract UIP Search}
An abstract UIP algorithm~\cite{DBLP:journals/fmsd/BrainDGHK14} 
traverses the trail $\trail$ starting from the conflict node and 
computes a cut that suffices to produce a conflict.  
For example, consider a formula $\formula:= \{x{+}4{=}z \wedge 
x{+}z{=}2y \wedge z{+}y>10\}$.  The ACG in Fig.~\ref{uip} records the 
sequence of deductions in the octagon domain that are inferred from 
a decision $(x{\leq}0)$ for the formula $\formula$.  The arrows (in red) 
shows the relationship between the reason trail and propagation trail in 
bottom of Fig.~\ref{uip}.  For the partial abstract value, 
$\absval=\{x\leq0 \wedge x+z\leq4 \wedge z\leq4\}$, 
obtained from the trail, the result of the abstract deduction transformer is
$\abstrans{\octagons}{y=(x+z)/2}(\absval)=\{x+y\leq2, y\leq2, y+z\leq6\}$.
A conflict ($\bot$) is reached for the decision ${x{\leq}0}$.   
Note that there exist multiple incomparable reasons for conflict,
marked as {\em cut0} and {\em cut1} in Fig.~\ref{uip}.  Here, cut0 is 
the first UIP (node closest to conflict node).  Choosing cut0 yields 
a learnt clause $(y+z>6)$, which is obtained by 
negating the reason for conflict.  The abstract UIP algorithm returns a learnt 
transformer $\aunit$, which is described next.\\ 
%    


\noindent \textbf{Learning in Template Polyhedra Domain}
%
Learning in a propositional solvers yields an asserting
clause~\cite{cdcl} that expresses the negation of the conflict
reasons.  We present a lattice-theoretic generalisation of the 
{\em unit rule} for template-based abstract domains that learns a new 
transformer called {\em abstract unit transformer} $(\aunit)$.    
We add $\aunit$ to the set of abstract transformers $\abstransset$. 
$\aunit$ is a generalisation of the propositional unit rule to
numerical domains.  For an abstract lattice $\domain$ with
complementable meet irreducibles and a set of meet irreducibles $\conflictset
\subseteq \domain$ such that $\bigsqcap
\conflictset$ does not satisfy $\formula$, $\aunit_\conflictset: \domain \rightarrow
\domain$ is formally defined as follows.
\[ \aunit_\conflictset(\absval) =
 \left\{\begin{array}{l@{\quad}l@{\qquad}l}
  \bot       & \text{if } \absval \sqsubseteq \bigsqcap \conflictset & (1)\\
  \bar{t}    & \text{if } t \in \conflictset \; \text{and} \; \forall t' \in \conflictset
  \setminus \{t\}. \absval  \sqsubseteq t' & (2) \\
  \top & \text{otherwise} & (3) \\
 \end{array}\right.
\]
Rule (1) shows $\aunit$ returns $\bot$ since 
$\absval \sqsubseteq \bigsqcap \conflictset$ is conflicting.  
Rule (2) of $\aunit$ infer a valid meet irreducible, 
which implies that $\conflictset$ is unit.  Rule (3) of  
$\aunit$ returns $\top$ which implies that the learnt clause is not 
{\em asserting} after backtracking.  This would prevent any new 
deductions from the learnt clause. Progress is then made by decisions. An 
example of $\aunit$ for $\conflictset = \{x \geq 2, x \leq 5, y \leq 7 \}$ is
shown below. \\
%
\Omit{
Let us consider an example, where $\conflictset = \{x \geq 2, x \leq 5, y
\leq 7 \}$ and $\absval = (x \geq 3\wedge\allowbreak x \leq
4\wedge\allowbreak y \geq 5\wedge y \leq 6)$.  Then
$\aunit_\conflictset(\absval) = \bot$ using the rule (1), since $\absval
\sqsubseteq \bigsqcap\conflictset$.  Now, consider another abstract value
$\absval = (x \geq 3\wedge x \leq 4)$ and the same $\conflictset$ as above,
then $\aunit_\conflictset(\absval) = (y \geq 8)$ using rule (2), since
$\absval \sqsubseteq (x \geq 2)$ and $\absval \sqsubseteq (x \leq 5)$. \\ 
}
Rule 1: 
%$\conflictset = \{x \geq 2, x \leq 5, y \leq 7 \}$ 
For $\absval = (x \geq 3\wedge\allowbreak x \leq
4\wedge\allowbreak y \geq 5\wedge y \leq 6)$, 
$\aunit_\conflictset(\absval) = \bot$, since $\absval
\sqsubseteq \bigsqcap\conflictset$.  \\ 
Rule 2:  
%$\conflictset = \{x \geq 2, x \leq 5, y \leq 7 \}$ and 
For $\absval = (x \geq 3\wedge x \leq 4)$,  
$\aunit_\conflictset(\absval) = (y \geq 8)$, since
$\absval \sqsubseteq (2 \leq x \leq 5)$. \\ 
%
Rule 3: 
%$\conflictset = \{x \geq 2, x \leq 5, y \leq 7 \}$ and 
For $\absval = (x \geq 1\wedge y \leq 10)$, 
$\aunit_\conflictset(\absval) = \top$. \\


\noindent \textbf{Backjumping}
A backjumping procedure removes all the meet irreducibles from 
the trail up to a decision level that restores the analysis to a
non-conflicting state.  The backjumping level is defined by the
meet irreducibles of the conflict clause that is closest 
to the root (decision level~0) where the conflict
clause is still unit.  If a conflict clause is globally unit, then the
backjumping level is the root of the search tree and
$\analyzeconflict$ returns $\false$, otherwise it returns $\true$.
