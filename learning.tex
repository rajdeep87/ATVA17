\section{Abstract Conflict Analysis for Template Polyhedra}\label{sec:conflict}
%
Propositional conflict analysis with FIRST-UIP~\cite{cdcl} can be seen as
abductive reasoning that under-approximates a set of models that do not
satisfy a formula~\cite{sas12,dhk2013-popl}.  Thus, we view abduction as 
De Morgan dual of deduction whose result don't require to be consistent with respect
to a background theory.  Below, we present an abstract
conflict analysis procedure, $\analyzeconflict$ of
Algorithm~\ref{Alg:acdcl}, that uses a domain-specific abductive transformer
for effective learning.  A~conflict analysis procedure involves two steps:
{\em abduction} and {\em heuristic choice for generalisation}.  Abduction
infers possible generalised reasons for a conflict which is followed by
heuristically selecting a generalisation.  Below, we define a global 
conflict transformer that gives a set of models that do not satisfy a 
formula.  
\begin{definition}
  Given a formula $\formula$, a downwards closed set of abstract elements $Q$,
  and domain $\domain$, $\conf{\formula}{\domain}(Q) = \{u \in \domain \mid u \in
  Q \vee u \not\models \formula \}$, that is, it returns 
  the set of abstract models that do not satisfy $\formula$ or are approximated by 
  $Q$.  An abstract abductive transformer, $\abd{\formula}{\domain}(Q)$, corresponds to 
  the under-approximation of the global conflict transformer,
  $\conf{\formula}{\domain}(Q)$.  
\end{definition}
%
%Informally, $\conf$ is a transformer over the underapproximate downset closure
%lattice of $D$.
%For formula $\formula$ and domain $\domain$, 
%
For example, given a formula $\formula=(x=y+1\; \wedge\; x\geq 0)$ and an interval 
abstract element $Q=(y\leq-5 \wedge x \leq -4)$, 
$\conf{\formula}{\intervals}(Q) = \{(y\leq-5 \land x \leq -4),\linebreak 
(y \leq -2 \land x < 0), \ldots, (y \leq 2 \land x \leq 10), \ldots \}$.  
%Informally, $\conf{\formula}{\intervals}$ computes the most general set of
%incomparable reasons under which $\formula$ implies $\bot$.
%the truth of $Q$.
%(or $\bot$ since $\formula$ is unsatisfiable under $a$). 
%
Now, an abstract abductive transformer for $\formula$ 
%=(x=y+1\; \wedge x \geq 0)$ 
is given by, $\abd{\formula}{\intervals}(Q)=(y \leq -2 \land x < 0)$, 
which clearly underapproximates $\conf{\formula}{\intervals}$ as well as 
strictly generalizes the reason for $Q$. 
%


The main idea of abductive  reasoning is to iteratively replace an abstract
element $s$ in the conflict reason by a partial assignment that is sufficient 
to infer $s$.  Conflict abduction is performed by obtaining cuts through markings 
in the trail $\trail$, by the application of abstract Unique Implication Point (UIP) 
search algorithm~\cite{cdcl}.  Every cut is a reason for conflict.  The UIP
search can be understood through graph cutting in an Abstract Conflict
Graph, which is defined next. 
\begin{definition}
  An Abstract Conflict Graph (ACG) is a directed acyclic graph in
  which the vertices are defined by deduced elements (including a
  special conflict node $(\bot)$) or a decision node in the trail
  $\trail$.  The edges in ACG are obtained from the reason trail
  $\reasons$ that maps pairs of elements in $\trail$ to the abstract
  transformers that are used to derive the deduced elements.  \Omit{
    if $\abstrans{\domain}{\constraint}(a) = u$, where $u$ is an
    unique deduction from $\abstrans{\domain}{\constraint}$ and $a$ is
    an initial abstract value, then there is a directed edge from each
    elements of $a$, whose variable set has non-empty intersection
    with the variable set of $\constraint$, to $u$.  }
\end{definition}
%
\begin{figure}[t]
\centering
\scalebox{.6}{\import{figures/}{oct_implgraph.pspdftex}}
\caption{\label{uip}Finding the Abstract UIP in the Octagon Domain}
\end{figure}  
%
\noindent \textbf{Abstract UIP Search}
An abstract UIP is a node in the ACG that must be traversed on every path between a 
decision node and the conflict. An abstract UIP cut is necessary to ensure that
the learnt clauses are asserting after backtracking and prevent cyclic algorithm
behavior. 
An abstract UIP algorithm~\cite{DBLP:journals/fmsd/BrainDGHK14} traverses
the trail $\trail$ starting from the conflict node and computes a cut that
suffices to produce a conflict.  For example, consider a formula $\formula:=
(x{+}4{=}z \wedge x{+}z{=}2y \wedge z{+}y{>}10)$.  As~before, the trail can be
viewed to represent an ACG, given in Fig.~\ref{uip}, that records the 
sequence of deductions in the octagon domain that are inferred
from a decision $x {\leq} 0$ for the formula $\formula$.  The arrows (in
red) indicate the relationship between the reason trail and propagation trail
in bottom of Fig.~\ref{uip}.  For the partial abstract value,
$\absval=(x\leq0 \wedge x+z\leq4 \wedge z\leq4)$, obtained from the trail,
the result of the abstract deduction transformer is
$\abstrans{\octagons}{y=(x+z)/2}(\absval)=(x+y\leq2 \wedge y\leq2 \wedge y+z\leq6)$. 
A~conflict ($\bot$) is reached for the decision ${x{\leq}0}$.  Note that
there exist multiple incomparable reasons for conflict, marked as {\em cut~0}
and {\em cut~1} in Fig.~\ref{uip}.  Here, cut~0 is the first UIP (node closest
to conflict node).  Choosing cut~0 yields a learnt clause $(y+z>6)$, which is
obtained by negating the reason for conflict.  The abstract UIP algorithm
returns a learnt transformer $\aunit$, which is described next.\\
%

\noindent \textbf{Learning in Template Polyhedra Domain}
%
Learning in a propositional solvers yields an asserting clause~\cite{cdcl}
that expresses the negation of the conflict reasons.  We present a
lattice-theoretic generalisation of the {\em unit rule} for template-based
abstract domains that learns a new transformer called {\em abstract unit
transformer} $(\aunit)$.  We add $\aunit$ to the set of abstract
transformers $\abstransset$.  $\aunit$ is a generalisation of the
propositional unit rule to numerical domains.  For an abstract lattice
$\domain$ with complementable meet irreducibles and a set of meet
irreducibles $\conflictset \subseteq \domain$ such that $\bigsqcap
\conflictset$ does not satisfy $\formula$, $\aunit_\conflictset: \domain
\rightarrow \domain$ is formally defined as follows.
%
\[ \aunit_\conflictset(\absval) =
 \left\{\begin{array}{l@{\quad}l@{\qquad}l}
  \bot       & \text{if } \absval \sqsubseteq \bigsqcap \conflictset & (1)\\
  \bar{t}    & \text{if } t \in \conflictset \; \text{and} \; \forall t' \in \conflictset
  \setminus \{t\}. \absval  \sqsubseteq t' & (2) \\
  \top & \text{otherwise} & (3) \\
 \end{array}\right.
\]
%
Rule (1) shows $\aunit$ returns $\bot$ since $\absval \sqsubseteq \bigsqcap
\conflictset$ is conflicting.  Rule (2) of $\aunit$ infers a valid meet
irreducible, which implies that $\conflictset$ is unit.  Rule (3) of
$\aunit$ returns $\top$ which implies that the learnt clause is not {\em
asserting} after backtracking.  This would prevent any new deductions from
the learnt clause.  Progress is then made by decisions.  An example of
$\aunit$ for $\conflictset = \{x \geq 2, x \leq 5, y \leq 7 \}$ is below.  \\
%
\Omit{
Let us consider an example, where $\conflictset = \{x \geq 2, x \leq 5, y
\leq 7 \}$ and $\absval = (x \geq 3\wedge\allowbreak x \leq
4\wedge\allowbreak y \geq 5\wedge y \leq 6)$.  Then
$\aunit_\conflictset(\absval) = \bot$ using the rule (1), since $\absval
\sqsubseteq \bigsqcap\conflictset$.  Now, consider another abstract value
$\absval = (x \geq 3\wedge x \leq 4)$ and the same $\conflictset$ as above,
then $\aunit_\conflictset(\absval) = (y \geq 8)$ using rule (2), since
$\absval \sqsubseteq (x \geq 2)$ and $\absval \sqsubseteq (x \leq 5)$. \\ 
}
Rule 1: 
%$\conflictset = \{x \geq 2, x \leq 5, y \leq 7 \}$ 
For $\absval = (x \geq 3\wedge\allowbreak x \leq
4\wedge\allowbreak y \geq 5\wedge y \leq 6)$, 
$\aunit_\conflictset(\absval) = \bot$, since $\absval
\sqsubseteq \bigsqcap\conflictset$.  \\ 
Rule 2:  
%$\conflictset = \{x \geq 2, x \leq 5, y \leq 7 \}$ and 
For $\absval = (x \geq 3\wedge x \leq 4)$,  
$\aunit_\conflictset(\absval) = (y \geq 8)$, since
$\absval \sqsubseteq (2 \leq x \leq 5)$. \\ 
%
Rule 3: 
%$\conflictset = \{x \geq 2, x \leq 5, y \leq 7 \}$ and 
For $\absval = (x \geq 1\wedge y \leq 10)$, 
$\aunit_\conflictset(\absval) = \top$. \\


\noindent \textbf{Backjumping}
A backjumping procedure removes all the meet irreducibles from 
the trail up to a decision level that restores the analysis to a
non-conflicting state.  The backjumping level is defined by the
meet irreducibles of the conflict clause that is closest 
to the root (decision level~0) where the conflict
clause is still unit.  If a conflict clause is globally unit, then the
backjumping level is the root of the search tree and
$\analyzeconflict$ returns $\false$, otherwise it returns $\true$.
