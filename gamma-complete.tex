\section{Gamma Complete Procedure in ACDLP}
\Omit{
The termination criterion of CDCL algorithm checks 
whether all variables in the CNF formula have been assigned, 
in which case the algorithm terminates indicating that the 
CNF formula is satisfiable~\cite{cdcl}. An alternative 
termination criterion is to check whether all clauses 
are satisfied.  However, modern CDCL based SAT solvers checks 
whether all the variables in the formula are assigned since the 
clause state can not be maintained accurately~\cite{cdcl}.
}

Recall that an over-approximating abstract transformer 
$f_{o}$ of a transformer $f$ is $\gamma$-complete with 
respect to an abstract element $a$ if $\gamma \circ f_{o}(a) = f \circ \gamma(a)$.  

If every transformer $f_{o}$ in $\varphi$ is $\gamma$-complete and 
produces a singleton value in the concrete, then ACDLP terminates 
with a satisfying assignment.  However, it is expensive to check whether all
$f_o$ are $\gamma$-complete.  An alternative is to check whether 
all program variables are assigned singleton values following the least fixed 
point computation in the propagation phase.  This corresponds to the 
termination condition in a SAT solver which checks whether all variables 
in the CNF formula have been assigned.

We present a more sophisticated $\gamma$-complete procedure in ACDLP based 
on a technique called {\em partial evaluation}.  A decision in ACDLP 
may render some $f_o \in \varphi$ vacuously true, for example, decisions on 
conditional branch predicates.  We call such decision, {\em choice decision}.
The core idea of partial evaluation technique is based on the fact 
that the variables appearing in non-vacuous $f_o$ can only contribute to 
determining the satisfiability of $\varphi$ following a choice decision. 
We call these variables {\em sensitive variables} $(V_s)$ and the rest 
{\em don't care variables} $(V_d)$.  Therefore, assignments to sensitive 
variables are sufficient to infer the truth of $\varphi$.  If $\varphi$ is 
{\em true}, then the abstract value obtained from the deductions via choice 
decision and sensitive variables produces a satisfying assignment for $\varphi$, 
and hence a counterexample. 

\Omit{
The partial evaluation technique identifies all non-singleton variables that 
do not contribute to a feasible path from a choice decision. We call these variables 
{\em don't care variables} for the current choice decision.  The don't care variables 
are identified via path-sensitive analysis that computes analysis information based on 
the predicates at conditional branch instructions. 
}

We explain the partial evaluation technique through don't care 
variables with a simple example.  Consider the program in 
Figure~\ref{fig:gc-example}. The second column of 
Figure~\ref{fig:gc-example} shows the corresponding SSA. 
The program is unsafe. The unsafe trace is shown in bold in third 
column of figure~\ref{fig:gc-example}. 
%
\begin{figure}[t]
\centering
\begin{tabular}{c|c|c}
\hline
Program & SSA & Unsafe trace \\
\hline
\scriptsize
\begin{lstlisting}[mathescape=true,language=C]
 int x, y; 
 _Bool c;
 assume(0 <= x && 
       x <= 1); 
 if(c) 
  x++; 
 else 
  x--; 
 assert(x<=0); 
\end{lstlisting} 
&
\begin{lstlisting}[mathescape=true,language=C]
cond#21=(x#16 >= 0 && 
        !(x#16 >= 2))
cond#22=(c#20=true)
guard#22=(cond#21 && guard#0)
x#23=1+x#16
guard#23=(cond#22 && guard#22)
cond#24=true
x#25=-1 + x#16
guard#25=(!cond#22 && guard#22)
x#phi26=(guard#25?x#25:x#23)
guard#26=(cond#24 && 
         guard#23 || guard#25)
!(x#phi26 > 0)||!guard#26
\end{lstlisting}
&
\begin{minipage}{3.0cm}
\centering
%\vspace*{0.3cm}
\scalebox{.5}{\import{figures/}{example.pspdftex}}
%\caption{\label{fig:oct}}
\end{minipage}
\\
\hline
\end{tabular}
\caption{\label{fig:gc-example}  C program, SSA and unsafe trace}
\end{figure}
%

Consider the SSA statement $x\#phi26 = (guard\#25 ? x\#25 : x\#23)$, shown in 
column~2 of Figure~\ref{fig:gc-example}. A CNF representation of the SSA 
statement is given by, 
$(!(guard\#25) \vee (x\#phi26=x\#25)) \wedge ((guard\#25) \vee (x\#phi26=x\#23))$. 
Let $C1=(!(guard\#25) \vee (x\#phi26 = x\#25))$ and $C2=((guard\#25) \vee
(x\#phi26=x\#23))$. 
Assume that we make a choice decision, $cond\#22=true$. This subsequently 
infers that $guard\#25=false$ which makes the clause $C1$ vacuously true. Hence, 
ACDLP only needs to evaluate the truth of $(x\#phi26=x\#23)$ in $C2$. 
Thus, the variable $x\#25$ is don't care variable since it belongs to the 
vacuous clause $C1$ obtained from the choice decision for conditional 
predicate $cond\#22=true$.  The set of sensitive variables are 
$\{x\#23, x\#phi26\}$.

When the don't care variables has already been identified, then ACDLP simply 
restricts the decisions to the {\em sensitive} variables.  We implemented a 
specific decision heuristic called {\em singleton decision heuristic} that 
explicitly assigns singleton values to these {\em sensitive variables}.  
For our example, ACDLP makes a decision 
on the interval of $x\#23 = [1,2]$, say $x\#23=1$, followed by the propagation 
which immediately infers that the program is unsafe.  ACDLP then terminates
and returns a satisfying assignment to $\varphi$ that contains concrete 
assignments to the sensitive variables only.

However, if the assignment to the sensitive variables does not satisfy 
$\varphi$, then ACDLP undoes all the deductions made from the singleton 
assignments to the sensitive variables and backtracks up to the point where 
the choice decision was made. 

\Omit{
The advantages of our proposed gamma-complete procedure are two folds. 
\begin{enumerate}
  \item It intelligently identifies the set of sensitive program variables through a 
choice decision thereby restricting future decisions on these variables.  
  \item It may terminate with a satisfying assignment (or counterexample trace) even 
when all variables in the program are not assigned, thereby leading to faster 
counterexample detection.
\end{enumerate}

%\section{Gamma Complete Procedure}
\begin{algorithm2e}
\DontPrintSemicolon
\SetKw{return}{return}
\SetKw{continue}{continue}
\SetKwRepeat{Do}{do}{while}
\SetKwData{conflict}{conflict}
\SetKwData{safe}{safe}
\SetKwData{sat}{sat}
\SetKwData{unsafe}{unsafe}
\SetKwData{unknown}{unknown}
\SetKwData{true}{true}
\SetKwData{false}{false}
\SetKwInOut{Input}{input}
\SetKwInOut{Output}{output}
\SetKwFor{Loop}{Loop}{}{}
\SetKw{KwNot}{not}
  \Input{Abstract value $\absval$ and set of sensitive variables $V_s$}
  \Output{\true if $\absval$ is gamma complete, else \false}
  \ForEach{$m \in \decomp(\absval)$}{
    \uIf{$vars(m) \in V_s$} {
      \uIf{$val(m, \absval)$ is singleton} {
        \continue
      }
      \uElse {
         \return \false \;
      }
    }
    \uElse {
      \continue 
    }
  }
  \return \true \;
\caption{Gamma Complete procedure $gammaComplete(\absval, V_s)$
  \label{Alg:gammacomplete}}
\end{algorithm2e}
%
Algorithm~\ref{Alg:gammacomplete} presents the gamma complete procedure that
checks whether a given abstract value $\absval$ is gamma complete with respect
to the given set of sensitive variables $V_s$. Note that the set $V_s$ is
obtained from the least fixed point computation in the deduction phase.  The
procedure $vars(m)$ returns the variables involved in the meet irreducibles $m$.
For example, $vars(x<10)$ returns $x$.  The procedure $val(m, \absval)$ returns 
the upper bound and the lower bound of a meet irreducible $m$ with respect to
the abstract value $\absval$.  For example, $val(x>5, x>5 \wedge y>10 \wedge x<10)$  
returns the interval $x=[6,9]$.  The procedure $vars()$ and $val()$ are domain
operations. 
}
%
\Omit{
\begin{mytheorem}
  (Soundness). Given a choice decision $d$, let $V_s$ be the set of sensitive 
  variables and $a$ be the abstract value obtained from deductions
  through $V_s$ and $d$. If $gammaComplete(a,V_s)$ returns {\em true}, then 
  $\varphi$ is satisfiable, and ACDLP terminates with the counterexample trace $a$.
\end{mytheorem}
}
%
